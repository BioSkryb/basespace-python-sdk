% Generated by Sphinx.
\def\sphinxdocclass{report}
\documentclass[letterpaper,10pt,english]{sphinxmanual}
\usepackage[utf8]{inputenc}
\DeclareUnicodeCharacter{00A0}{\nobreakspace}
\usepackage[T1]{fontenc}
\usepackage{babel}
\usepackage{times}
\usepackage[Bjarne]{fncychap}
\usepackage{longtable}
\usepackage{sphinx}
\usepackage{multirow}


\title{BaseSpacePy Documentation}
\date{October 02, 2012}
\release{0.1.2}
\author{Morten Kallberg}
\newcommand{\sphinxlogo}{}
\renewcommand{\releasename}{Release}
\makeindex

\makeatletter
\def\PYG@reset{\let\PYG@it=\relax \let\PYG@bf=\relax%
    \let\PYG@ul=\relax \let\PYG@tc=\relax%
    \let\PYG@bc=\relax \let\PYG@ff=\relax}
\def\PYG@tok#1{\csname PYG@tok@#1\endcsname}
\def\PYG@toks#1+{\ifx\relax#1\empty\else%
    \PYG@tok{#1}\expandafter\PYG@toks\fi}
\def\PYG@do#1{\PYG@bc{\PYG@tc{\PYG@ul{%
    \PYG@it{\PYG@bf{\PYG@ff{#1}}}}}}}
\def\PYG#1#2{\PYG@reset\PYG@toks#1+\relax+\PYG@do{#2}}

\def\PYG@tok@gd{\def\PYG@tc##1{\textcolor[rgb]{0.63,0.00,0.00}{##1}}}
\def\PYG@tok@gu{\let\PYG@bf=\textbf\def\PYG@tc##1{\textcolor[rgb]{0.50,0.00,0.50}{##1}}}
\def\PYG@tok@gt{\def\PYG@tc##1{\textcolor[rgb]{0.00,0.25,0.82}{##1}}}
\def\PYG@tok@gs{\let\PYG@bf=\textbf}
\def\PYG@tok@gr{\def\PYG@tc##1{\textcolor[rgb]{1.00,0.00,0.00}{##1}}}
\def\PYG@tok@cm{\let\PYG@it=\textit\def\PYG@tc##1{\textcolor[rgb]{0.25,0.50,0.56}{##1}}}
\def\PYG@tok@vg{\def\PYG@tc##1{\textcolor[rgb]{0.73,0.38,0.84}{##1}}}
\def\PYG@tok@m{\def\PYG@tc##1{\textcolor[rgb]{0.13,0.50,0.31}{##1}}}
\def\PYG@tok@mh{\def\PYG@tc##1{\textcolor[rgb]{0.13,0.50,0.31}{##1}}}
\def\PYG@tok@cs{\def\PYG@tc##1{\textcolor[rgb]{0.25,0.50,0.56}{##1}}\def\PYG@bc##1{\colorbox[rgb]{1.00,0.94,0.94}{##1}}}
\def\PYG@tok@ge{\let\PYG@it=\textit}
\def\PYG@tok@vc{\def\PYG@tc##1{\textcolor[rgb]{0.73,0.38,0.84}{##1}}}
\def\PYG@tok@il{\def\PYG@tc##1{\textcolor[rgb]{0.13,0.50,0.31}{##1}}}
\def\PYG@tok@go{\def\PYG@tc##1{\textcolor[rgb]{0.19,0.19,0.19}{##1}}}
\def\PYG@tok@cp{\def\PYG@tc##1{\textcolor[rgb]{0.00,0.44,0.13}{##1}}}
\def\PYG@tok@gi{\def\PYG@tc##1{\textcolor[rgb]{0.00,0.63,0.00}{##1}}}
\def\PYG@tok@gh{\let\PYG@bf=\textbf\def\PYG@tc##1{\textcolor[rgb]{0.00,0.00,0.50}{##1}}}
\def\PYG@tok@ni{\let\PYG@bf=\textbf\def\PYG@tc##1{\textcolor[rgb]{0.84,0.33,0.22}{##1}}}
\def\PYG@tok@nl{\let\PYG@bf=\textbf\def\PYG@tc##1{\textcolor[rgb]{0.00,0.13,0.44}{##1}}}
\def\PYG@tok@nn{\let\PYG@bf=\textbf\def\PYG@tc##1{\textcolor[rgb]{0.05,0.52,0.71}{##1}}}
\def\PYG@tok@no{\def\PYG@tc##1{\textcolor[rgb]{0.38,0.68,0.84}{##1}}}
\def\PYG@tok@na{\def\PYG@tc##1{\textcolor[rgb]{0.25,0.44,0.63}{##1}}}
\def\PYG@tok@nb{\def\PYG@tc##1{\textcolor[rgb]{0.00,0.44,0.13}{##1}}}
\def\PYG@tok@nc{\let\PYG@bf=\textbf\def\PYG@tc##1{\textcolor[rgb]{0.05,0.52,0.71}{##1}}}
\def\PYG@tok@nd{\let\PYG@bf=\textbf\def\PYG@tc##1{\textcolor[rgb]{0.33,0.33,0.33}{##1}}}
\def\PYG@tok@ne{\def\PYG@tc##1{\textcolor[rgb]{0.00,0.44,0.13}{##1}}}
\def\PYG@tok@nf{\def\PYG@tc##1{\textcolor[rgb]{0.02,0.16,0.49}{##1}}}
\def\PYG@tok@si{\let\PYG@it=\textit\def\PYG@tc##1{\textcolor[rgb]{0.44,0.63,0.82}{##1}}}
\def\PYG@tok@s2{\def\PYG@tc##1{\textcolor[rgb]{0.25,0.44,0.63}{##1}}}
\def\PYG@tok@vi{\def\PYG@tc##1{\textcolor[rgb]{0.73,0.38,0.84}{##1}}}
\def\PYG@tok@nt{\let\PYG@bf=\textbf\def\PYG@tc##1{\textcolor[rgb]{0.02,0.16,0.45}{##1}}}
\def\PYG@tok@nv{\def\PYG@tc##1{\textcolor[rgb]{0.73,0.38,0.84}{##1}}}
\def\PYG@tok@s1{\def\PYG@tc##1{\textcolor[rgb]{0.25,0.44,0.63}{##1}}}
\def\PYG@tok@gp{\let\PYG@bf=\textbf\def\PYG@tc##1{\textcolor[rgb]{0.78,0.36,0.04}{##1}}}
\def\PYG@tok@sh{\def\PYG@tc##1{\textcolor[rgb]{0.25,0.44,0.63}{##1}}}
\def\PYG@tok@ow{\let\PYG@bf=\textbf\def\PYG@tc##1{\textcolor[rgb]{0.00,0.44,0.13}{##1}}}
\def\PYG@tok@sx{\def\PYG@tc##1{\textcolor[rgb]{0.78,0.36,0.04}{##1}}}
\def\PYG@tok@bp{\def\PYG@tc##1{\textcolor[rgb]{0.00,0.44,0.13}{##1}}}
\def\PYG@tok@c1{\let\PYG@it=\textit\def\PYG@tc##1{\textcolor[rgb]{0.25,0.50,0.56}{##1}}}
\def\PYG@tok@kc{\let\PYG@bf=\textbf\def\PYG@tc##1{\textcolor[rgb]{0.00,0.44,0.13}{##1}}}
\def\PYG@tok@c{\let\PYG@it=\textit\def\PYG@tc##1{\textcolor[rgb]{0.25,0.50,0.56}{##1}}}
\def\PYG@tok@mf{\def\PYG@tc##1{\textcolor[rgb]{0.13,0.50,0.31}{##1}}}
\def\PYG@tok@err{\def\PYG@bc##1{\fcolorbox[rgb]{1.00,0.00,0.00}{1,1,1}{##1}}}
\def\PYG@tok@kd{\let\PYG@bf=\textbf\def\PYG@tc##1{\textcolor[rgb]{0.00,0.44,0.13}{##1}}}
\def\PYG@tok@ss{\def\PYG@tc##1{\textcolor[rgb]{0.32,0.47,0.09}{##1}}}
\def\PYG@tok@sr{\def\PYG@tc##1{\textcolor[rgb]{0.14,0.33,0.53}{##1}}}
\def\PYG@tok@mo{\def\PYG@tc##1{\textcolor[rgb]{0.13,0.50,0.31}{##1}}}
\def\PYG@tok@mi{\def\PYG@tc##1{\textcolor[rgb]{0.13,0.50,0.31}{##1}}}
\def\PYG@tok@kn{\let\PYG@bf=\textbf\def\PYG@tc##1{\textcolor[rgb]{0.00,0.44,0.13}{##1}}}
\def\PYG@tok@o{\def\PYG@tc##1{\textcolor[rgb]{0.40,0.40,0.40}{##1}}}
\def\PYG@tok@kr{\let\PYG@bf=\textbf\def\PYG@tc##1{\textcolor[rgb]{0.00,0.44,0.13}{##1}}}
\def\PYG@tok@s{\def\PYG@tc##1{\textcolor[rgb]{0.25,0.44,0.63}{##1}}}
\def\PYG@tok@kp{\def\PYG@tc##1{\textcolor[rgb]{0.00,0.44,0.13}{##1}}}
\def\PYG@tok@w{\def\PYG@tc##1{\textcolor[rgb]{0.73,0.73,0.73}{##1}}}
\def\PYG@tok@kt{\def\PYG@tc##1{\textcolor[rgb]{0.56,0.13,0.00}{##1}}}
\def\PYG@tok@sc{\def\PYG@tc##1{\textcolor[rgb]{0.25,0.44,0.63}{##1}}}
\def\PYG@tok@sb{\def\PYG@tc##1{\textcolor[rgb]{0.25,0.44,0.63}{##1}}}
\def\PYG@tok@k{\let\PYG@bf=\textbf\def\PYG@tc##1{\textcolor[rgb]{0.00,0.44,0.13}{##1}}}
\def\PYG@tok@se{\let\PYG@bf=\textbf\def\PYG@tc##1{\textcolor[rgb]{0.25,0.44,0.63}{##1}}}
\def\PYG@tok@sd{\let\PYG@it=\textit\def\PYG@tc##1{\textcolor[rgb]{0.25,0.44,0.63}{##1}}}

\def\PYGZbs{\char`\\}
\def\PYGZus{\char`\_}
\def\PYGZob{\char`\{}
\def\PYGZcb{\char`\}}
\def\PYGZca{\char`\^}
\def\PYGZsh{\char`\#}
\def\PYGZpc{\char`\%}
\def\PYGZdl{\char`\$}
\def\PYGZti{\char`\~}
% for compatibility with earlier versions
\def\PYGZat{@}
\def\PYGZlb{[}
\def\PYGZrb{]}
\makeatother

\begin{document}

\maketitle
\tableofcontents
\phantomsection\label{index::doc}



\chapter{Available modules}
\label{Available modules:available-modules}\label{Available modules::doc}\label{Available modules:basespacepy}

\section{API}
\label{Available modules:api}\index{BaseSpaceAPI (class in BaseSpacePy.api.BaseSpaceAPI)}

\begin{fulllineitems}
\phantomsection\label{Available modules:BaseSpacePy.api.BaseSpaceAPI.BaseSpaceAPI}\pysiglinewithargsret{\strong{class }\code{BaseSpacePy.api.BaseSpaceAPI.}\bfcode{BaseSpaceAPI}}{\emph{clientKey}, \emph{clientSecret}, \emph{apiServer}, \emph{version}, \emph{appSessionId}, \emph{AccessToken='`}}{}
The main API class used for all communication with with the REST server
\index{appResultFileUpload() (BaseSpacePy.api.BaseSpaceAPI.BaseSpaceAPI method)}

\begin{fulllineitems}
\phantomsection\label{Available modules:BaseSpacePy.api.BaseSpaceAPI.BaseSpaceAPI.appResultFileUpload}\pysiglinewithargsret{\bfcode{appResultFileUpload}}{\emph{Id}, \emph{localPath}, \emph{fileName}, \emph{directory}, \emph{contentType}, \emph{multipart=0}}{}
Uploads a file associated with an AppResult to BaseSpace and returns the corresponding file object
\begin{quote}\begin{description}
\item[{Parameters}] \leavevmode\begin{itemize}
\item {} 
\textbf{Id} -- AppResult id.

\item {} 
\textbf{localPath} -- The local path to the file to be uploaded.

\item {} 
\textbf{fileName} -- The desired filename in the AppResult folder on the BaseSpace server.

\item {} 
\textbf{directory} -- The directory the file should be placed in.

\item {} 
\textbf{contentType} -- The content-type of the file.

\end{itemize}

\end{description}\end{quote}

\end{fulllineitems}

\index{createAppResult() (BaseSpacePy.api.BaseSpaceAPI.BaseSpaceAPI method)}

\begin{fulllineitems}
\phantomsection\label{Available modules:BaseSpacePy.api.BaseSpaceAPI.BaseSpaceAPI.createAppResult}\pysiglinewithargsret{\bfcode{createAppResult}}{\emph{Id}, \emph{name}, \emph{desc}, \emph{samples=}\optional{}, \emph{appSessionId=None}}{}
Create an AppResult object
\begin{quote}\begin{description}
\item[{Parameters}] \leavevmode\begin{itemize}
\item {} 
\textbf{Id} -- The id of the project in which the AppResult is to be added

\item {} 
\textbf{name} -- The name of the AppResult

\item {} 
\textbf{desc} -- A describtion of the AppResult

\item {} 
\textbf{samples} -- (Optional) The samples

\item {} 
\textbf{appSessionId} -- (Optional) If no appSessionId is given, the id used to initialize the BaseSpaceAPI instance

\end{itemize}

\end{description}\end{quote}

will be used. If appSessionId is set equal to an empty string, a new appsession will be created for the

\end{fulllineitems}

\index{fileDownload() (BaseSpacePy.api.BaseSpaceAPI.BaseSpaceAPI method)}

\begin{fulllineitems}
\phantomsection\label{Available modules:BaseSpacePy.api.BaseSpaceAPI.BaseSpaceAPI.fileDownload}\pysiglinewithargsret{\bfcode{fileDownload}}{\emph{Id}, \emph{localDir}, \emph{name}, \emph{range=}\optional{}}{}
Downloads a BaseSpace file to a local directory
\begin{quote}\begin{description}
\item[{Parameters}] \leavevmode\begin{itemize}
\item {} 
\textbf{Id} -- The file id

\item {} 
\textbf{localDir} -- The local directory to place the file in

\item {} 
\textbf{name} -- The name of the local file

\item {} 
\textbf{range} -- (Optional) The byte range of the file to retrieve (not yet implemented)

\end{itemize}

\end{description}\end{quote}

\end{fulllineitems}

\index{filterVariantSet() (BaseSpacePy.api.BaseSpaceAPI.BaseSpaceAPI method)}

\begin{fulllineitems}
\phantomsection\label{Available modules:BaseSpacePy.api.BaseSpaceAPI.BaseSpaceAPI.filterVariantSet}\pysiglinewithargsret{\bfcode{filterVariantSet}}{\emph{Id}, \emph{Chrom}, \emph{StartPos}, \emph{EndPos}, \emph{Format}, \emph{queryPars=\{`Limit': `100'}, \emph{`SortBy': `Position'}, \emph{`SortDir': `Asc'}, \emph{`Offset': `0'\}}}{}
List the variants in a set of variants. Maximum returned records is 1000
\begin{quote}\begin{description}
\item[{Parameters}] \leavevmode\begin{itemize}
\item {} 
\textbf{Id} -- The id of the variant file

\item {} 
\textbf{Chrom} -- The chromosome of interest

\item {} 
\textbf{StartPos} -- The start position of the sequence of interest

\item {} 
\textbf{EndPos} -- The start position of the sequence of interest

\item {} 
\textbf{Format} -- Set to `vcf' to get the results as lines in VCF format

\item {} 
\textbf{queryPars} -- An (optional) object of type QueryParameters for custom sorting and filtering

\end{itemize}

\end{description}\end{quote}

\end{fulllineitems}

\index{getAccess() (BaseSpacePy.api.BaseSpaceAPI.BaseSpaceAPI method)}

\begin{fulllineitems}
\phantomsection\label{Available modules:BaseSpacePy.api.BaseSpaceAPI.BaseSpaceAPI.getAccess}\pysiglinewithargsret{\bfcode{getAccess}}{\emph{obj}, \emph{accessType='write'}, \emph{web=0}, \emph{redirectURL='`}, \emph{state='`}}{}~\begin{quote}\begin{description}
\item[{Parameters}] \leavevmode\begin{itemize}
\item {} 
\textbf{obj} -- The data object we wish to get access to

\item {} 
\textbf{accessType} -- (Optional) the type of access (read\textbar{}write), default is write

\item {} 
\textbf{web} -- (Optional) true if the App is web-based, default is false meaning a device based app

\item {} 
\textbf{redirectURL} -- (Optional) For the web-based case, a

\item {} 
\textbf{state} -- (Optional)

\end{itemize}

\end{description}\end{quote}

\end{fulllineitems}

\index{getAccessToken() (BaseSpacePy.api.BaseSpaceAPI.BaseSpaceAPI method)}

\begin{fulllineitems}
\phantomsection\label{Available modules:BaseSpacePy.api.BaseSpaceAPI.BaseSpaceAPI.getAccessToken}\pysiglinewithargsret{\bfcode{getAccessToken}}{}{}
Returns the access-token that was used to initialize the BaseSpaceAPI object.

\end{fulllineitems}

\index{getAccessibleRunsByUser() (BaseSpacePy.api.BaseSpaceAPI.BaseSpaceAPI method)}

\begin{fulllineitems}
\phantomsection\label{Available modules:BaseSpacePy.api.BaseSpaceAPI.BaseSpaceAPI.getAccessibleRunsByUser}\pysiglinewithargsret{\bfcode{getAccessibleRunsByUser}}{\emph{Id}, \emph{queryPars=\{`Limit': `100'}, \emph{`SortBy': `Id'}, \emph{`SortDir': `Asc'}, \emph{`Offset': `0'\}}}{}
Returns a list of accessible runs for the User with id=Id
\begin{quote}\begin{description}
\item[{Parameters}] \leavevmode\begin{itemize}
\item {} 
\textbf{Id} -- An user id

\item {} 
\textbf{queryPars} -- An (optional) object of type QueryParameters for custom sorting and filtering

\end{itemize}

\end{description}\end{quote}

\end{fulllineitems}

\index{getAppResultById() (BaseSpacePy.api.BaseSpaceAPI.BaseSpaceAPI method)}

\begin{fulllineitems}
\phantomsection\label{Available modules:BaseSpacePy.api.BaseSpaceAPI.BaseSpaceAPI.getAppResultById}\pysiglinewithargsret{\bfcode{getAppResultById}}{\emph{Id}}{}
Returns an AppResult object corresponding to Id
\begin{quote}\begin{description}
\item[{Parameters}] \leavevmode
\textbf{Id} -- The Id of the AppResult

\end{description}\end{quote}

\end{fulllineitems}

\index{getAppResultFiles() (BaseSpacePy.api.BaseSpaceAPI.BaseSpaceAPI method)}

\begin{fulllineitems}
\phantomsection\label{Available modules:BaseSpacePy.api.BaseSpaceAPI.BaseSpaceAPI.getAppResultFiles}\pysiglinewithargsret{\bfcode{getAppResultFiles}}{\emph{Id}, \emph{queryPars=\{`Limit': `100'}, \emph{`SortBy': `Id'}, \emph{`SortDir': `Asc'}, \emph{`Offset': `0'\}}}{}
Returns a list of File object for the AppResult with id  = Id
\begin{quote}\begin{description}
\item[{Parameters}] \leavevmode\begin{itemize}
\item {} 
\textbf{Id} -- The id of the appresult.

\item {} 
\textbf{queryPars} -- An (optional) object of type QueryParameters for custom sorting and filtering

\end{itemize}

\end{description}\end{quote}

\end{fulllineitems}

\index{getAppResultsByProject() (BaseSpacePy.api.BaseSpaceAPI.BaseSpaceAPI method)}

\begin{fulllineitems}
\phantomsection\label{Available modules:BaseSpacePy.api.BaseSpaceAPI.BaseSpaceAPI.getAppResultsByProject}\pysiglinewithargsret{\bfcode{getAppResultsByProject}}{\emph{Id}, \emph{queryPars=\{`Limit': `100'}, \emph{`SortBy': `Id'}, \emph{`SortDir': `Asc'}, \emph{`Offset': `0'\}}, \emph{statuses=}\optional{}}{}
Returns a list of AppResult object associated with the project with Id
\begin{quote}\begin{description}
\item[{Parameters}] \leavevmode\begin{itemize}
\item {} 
\textbf{Id} -- The project id

\item {} 
\textbf{queryPars} -- An (optional) object of type QueryParameters for custom sorting and filtering

\item {} 
\textbf{statuses} -- An (optional) list of AppResult statuses to filter by

\end{itemize}

\end{description}\end{quote}

\end{fulllineitems}

\index{getAppSession() (BaseSpacePy.api.BaseSpaceAPI.BaseSpaceAPI method)}

\begin{fulllineitems}
\phantomsection\label{Available modules:BaseSpacePy.api.BaseSpaceAPI.BaseSpaceAPI.getAppSession}\pysiglinewithargsret{\bfcode{getAppSession}}{\emph{Id='`}}{}
Returns an AppSession instance containing user and data-type the app was triggered by/on 
:param Id: (Optional) The AppSessionId, id not supplied the AppSessionId used for instantiating
the BaseSpaceAPI instance.
\begin{quote}\begin{description}
\item[{Parameters}] \leavevmode
\textbf{Id} -- (Optional) AppSession id, if not supplied the AppSession id used to initialize the

\end{description}\end{quote}

\end{fulllineitems}

\index{getAppSessionById() (BaseSpacePy.api.BaseSpaceAPI.BaseSpaceAPI method)}

\begin{fulllineitems}
\phantomsection\label{Available modules:BaseSpacePy.api.BaseSpaceAPI.BaseSpaceAPI.getAppSessionById}\pysiglinewithargsret{\bfcode{getAppSessionById}}{\emph{id}}{}
Returns the appSession identified by id
\begin{quote}\begin{description}
\item[{Parameters}] \leavevmode
\textbf{id} -- The id of the appSession

\end{description}\end{quote}

\end{fulllineitems}

\index{getAvailableGenomes() (BaseSpacePy.api.BaseSpaceAPI.BaseSpaceAPI method)}

\begin{fulllineitems}
\phantomsection\label{Available modules:BaseSpacePy.api.BaseSpaceAPI.BaseSpaceAPI.getAvailableGenomes}\pysiglinewithargsret{\bfcode{getAvailableGenomes}}{\emph{queryPars=\{`Limit': `100'}, \emph{`SortBy': `Id'}, \emph{`SortDir': `Asc'}, \emph{`Offset': `0'\}}}{}
Returns a list of all available genomes
\begin{quote}\begin{description}
\item[{Parameters}] \leavevmode
\textbf{queryPars} -- An (optional) object of type QueryParameters for custom sorting and filtering

\end{description}\end{quote}

\end{fulllineitems}

\index{getCoverageMetaInfo() (BaseSpacePy.api.BaseSpaceAPI.BaseSpaceAPI method)}

\begin{fulllineitems}
\phantomsection\label{Available modules:BaseSpacePy.api.BaseSpaceAPI.BaseSpaceAPI.getCoverageMetaInfo}\pysiglinewithargsret{\bfcode{getCoverageMetaInfo}}{\emph{Id}, \emph{Chrom}}{}
Returns Metadata about coverage as a CoverageMetadata instance
\begin{quote}\begin{description}
\item[{Parameters}] \leavevmode\begin{itemize}
\item {} 
\textbf{Id} -- he Id of the Bam file

\item {} 
\textbf{Chrom} -- Chromosome to query

\end{itemize}

\end{description}\end{quote}

\end{fulllineitems}

\index{getFileById() (BaseSpacePy.api.BaseSpaceAPI.BaseSpaceAPI method)}

\begin{fulllineitems}
\phantomsection\label{Available modules:BaseSpacePy.api.BaseSpaceAPI.BaseSpaceAPI.getFileById}\pysiglinewithargsret{\bfcode{getFileById}}{\emph{Id}}{}
Returns a file object by Id
\begin{quote}\begin{description}
\item[{Parameters}] \leavevmode
\textbf{Id} -- The id of the file

\end{description}\end{quote}

\end{fulllineitems}

\index{getFilesBySample() (BaseSpacePy.api.BaseSpaceAPI.BaseSpaceAPI method)}

\begin{fulllineitems}
\phantomsection\label{Available modules:BaseSpacePy.api.BaseSpaceAPI.BaseSpaceAPI.getFilesBySample}\pysiglinewithargsret{\bfcode{getFilesBySample}}{\emph{Id}, \emph{queryPars=\{`Limit': `100'}, \emph{`SortBy': `Id'}, \emph{`SortDir': `Asc'}, \emph{`Offset': `0'\}}}{}
Returns a list of File objects associated with sample with Id
\begin{quote}\begin{description}
\item[{Parameters}] \leavevmode\begin{itemize}
\item {} 
\textbf{Id} -- A Sample id

\item {} 
\textbf{queryPars} -- An (optional) object of type QueryParameters for custom sorting and filtering

\end{itemize}

\end{description}\end{quote}

\end{fulllineitems}

\index{getGenomeById() (BaseSpacePy.api.BaseSpaceAPI.BaseSpaceAPI method)}

\begin{fulllineitems}
\phantomsection\label{Available modules:BaseSpacePy.api.BaseSpaceAPI.BaseSpaceAPI.getGenomeById}\pysiglinewithargsret{\bfcode{getGenomeById}}{\emph{Id}}{}
Returns an instance of Genome with the specified Id
\begin{quote}\begin{description}
\item[{Parameters}] \leavevmode
\textbf{Id} -- The genome id

\end{description}\end{quote}

\end{fulllineitems}

\index{getIntervalCoverage() (BaseSpacePy.api.BaseSpaceAPI.BaseSpaceAPI method)}

\begin{fulllineitems}
\phantomsection\label{Available modules:BaseSpacePy.api.BaseSpaceAPI.BaseSpaceAPI.getIntervalCoverage}\pysiglinewithargsret{\bfcode{getIntervalCoverage}}{\emph{Id}, \emph{Chrom}, \emph{StartPos=None}, \emph{EndPos=None}}{}
Mean coverage levels over a sequence interval
\begin{quote}\begin{description}
\item[{Parameters}] \leavevmode\begin{itemize}
\item {} 
\textbf{Id} -- Chromosome to query

\item {} 
\textbf{Chrom} -- The Id of the resource

\item {} 
\textbf{StartPos} -- Get coverage starting at this position. Default is 1

\item {} 
\textbf{EndPos} -- Get coverage up to and including this position. Default is StartPos + 1280

\end{itemize}

\end{description}\end{quote}

:return:CoverageResponse -- an instance of CoverageResponse

\end{fulllineitems}

\index{getProjectById() (BaseSpacePy.api.BaseSpaceAPI.BaseSpaceAPI method)}

\begin{fulllineitems}
\phantomsection\label{Available modules:BaseSpacePy.api.BaseSpaceAPI.BaseSpaceAPI.getProjectById}\pysiglinewithargsret{\bfcode{getProjectById}}{\emph{Id}}{}
Request a project object by Id
\begin{quote}\begin{description}
\item[{Parameters}] \leavevmode
\textbf{Id} -- The Id of the project

\end{description}\end{quote}

\end{fulllineitems}

\index{getProjectByUser() (BaseSpacePy.api.BaseSpaceAPI.BaseSpaceAPI method)}

\begin{fulllineitems}
\phantomsection\label{Available modules:BaseSpacePy.api.BaseSpaceAPI.BaseSpaceAPI.getProjectByUser}\pysiglinewithargsret{\bfcode{getProjectByUser}}{\emph{Id}, \emph{queryPars=\{`Limit': `100'}, \emph{`SortBy': `Id'}, \emph{`SortDir': `Asc'}, \emph{`Offset': `0'\}}}{}
Returns a list available projects for a User with the specified Id
\begin{quote}\begin{description}
\item[{Parameters}] \leavevmode\begin{itemize}
\item {} 
\textbf{Id} -- The id of the user

\item {} 
\textbf{qp} -- An (optional) object of type QueryParameters for custom sorting and filtering

\end{itemize}

\end{description}\end{quote}

\end{fulllineitems}

\index{getSampleById() (BaseSpacePy.api.BaseSpaceAPI.BaseSpaceAPI method)}

\begin{fulllineitems}
\phantomsection\label{Available modules:BaseSpacePy.api.BaseSpaceAPI.BaseSpaceAPI.getSampleById}\pysiglinewithargsret{\bfcode{getSampleById}}{\emph{Id}}{}
Returns a Sample object
\begin{quote}\begin{description}
\item[{Parameters}] \leavevmode
\textbf{Id} -- The id of the sample

\end{description}\end{quote}

\end{fulllineitems}

\index{getSamplesByProject() (BaseSpacePy.api.BaseSpaceAPI.BaseSpaceAPI method)}

\begin{fulllineitems}
\phantomsection\label{Available modules:BaseSpacePy.api.BaseSpaceAPI.BaseSpaceAPI.getSamplesByProject}\pysiglinewithargsret{\bfcode{getSamplesByProject}}{\emph{Id}, \emph{queryPars=\{`Limit': `100'}, \emph{`SortBy': `Id'}, \emph{`SortDir': `Asc'}, \emph{`Offset': `0'\}}}{}
Returns a list of samples associated with a project with Id
\begin{quote}\begin{description}
\item[{Parameters}] \leavevmode\begin{itemize}
\item {} 
\textbf{Id} -- The id of the project

\item {} 
\textbf{queryPars} -- An (optional) object of type QueryParameters for custom sorting and filtering

\end{itemize}

\end{description}\end{quote}

\end{fulllineitems}

\index{getServerUri() (BaseSpacePy.api.BaseSpaceAPI.BaseSpaceAPI method)}

\begin{fulllineitems}
\phantomsection\label{Available modules:BaseSpacePy.api.BaseSpaceAPI.BaseSpaceAPI.getServerUri}\pysiglinewithargsret{\bfcode{getServerUri}}{}{}
Returns the server uri used by this instance

\end{fulllineitems}

\index{getUserById() (BaseSpacePy.api.BaseSpaceAPI.BaseSpaceAPI method)}

\begin{fulllineitems}
\phantomsection\label{Available modules:BaseSpacePy.api.BaseSpaceAPI.BaseSpaceAPI.getUserById}\pysiglinewithargsret{\bfcode{getUserById}}{\emph{Id}}{}
Returns the User object corresponding to Id
\begin{quote}\begin{description}
\item[{Parameters}] \leavevmode
\textbf{Id} -- The Id of the user

\end{description}\end{quote}

\end{fulllineitems}

\index{getVariantMetadata() (BaseSpacePy.api.BaseSpaceAPI.BaseSpaceAPI method)}

\begin{fulllineitems}
\phantomsection\label{Available modules:BaseSpacePy.api.BaseSpaceAPI.BaseSpaceAPI.getVariantMetadata}\pysiglinewithargsret{\bfcode{getVariantMetadata}}{\emph{Id}, \emph{Format}}{}
Returns a VariantMetadata object for the variant file
\begin{quote}\begin{description}
\item[{Parameters}] \leavevmode\begin{itemize}
\item {} 
\textbf{Id} -- The Id of the VCF file

\item {} 
\textbf{Format} -- Set to `vcf' to get the results as lines in VCF format

\end{itemize}

\end{description}\end{quote}

\end{fulllineitems}

\index{getVerificationCode() (BaseSpacePy.api.BaseSpaceAPI.BaseSpaceAPI method)}

\begin{fulllineitems}
\phantomsection\label{Available modules:BaseSpacePy.api.BaseSpaceAPI.BaseSpaceAPI.getVerificationCode}\pysiglinewithargsret{\bfcode{getVerificationCode}}{\emph{scope}}{}
Returns the BaseSpace dictionary containing the verification code and verification url for the user to approve
access to a specific data scope.

Corresponding curl call:
curlCall = `curl -d ``response\_type=device\_code'' -d ``client\_id=' + client\_key + `'' -d ``scope=' + scope + `'' ` + deviceURL

For details see:
\href{https://developer.basespace.illumina.com/docs/content/documentation/authentication/obtaining-access-tokens}{https://developer.basespace.illumina.com/docs/content/documentation/authentication/obtaining-access-tokens}
\begin{quote}\begin{description}
\item[{Parameters}] \leavevmode
\textbf{scope} -- The scope that access is requested for

\end{description}\end{quote}

\end{fulllineitems}

\index{getWebVerificationCode() (BaseSpacePy.api.BaseSpaceAPI.BaseSpaceAPI method)}

\begin{fulllineitems}
\phantomsection\label{Available modules:BaseSpacePy.api.BaseSpaceAPI.BaseSpaceAPI.getWebVerificationCode}\pysiglinewithargsret{\bfcode{getWebVerificationCode}}{\emph{scope}, \emph{redirectURL}, \emph{state='`}}{}
Generates the URL the user should be redirected to for web-based authentication
\begin{quote}\begin{description}
\item[{Parameters}] \leavevmode\begin{itemize}
\item {} 
\textbf{scope} -- The scope that access is requested for

\item {} 
\textbf{redirectURL} -- The redirect URL

\end{itemize}

\item[{State }] \leavevmode
An optional state parameter that will passed through to the redirect response

\end{description}\end{quote}

\end{fulllineitems}

\index{obtainAccessToken() (BaseSpacePy.api.BaseSpaceAPI.BaseSpaceAPI method)}

\begin{fulllineitems}
\phantomsection\label{Available modules:BaseSpacePy.api.BaseSpaceAPI.BaseSpaceAPI.obtainAccessToken}\pysiglinewithargsret{\bfcode{obtainAccessToken}}{\emph{deviceCode}}{}
Returns a user specific access token.
\begin{quote}\begin{description}
\item[{Parameters}] \leavevmode
\textbf{deviceCode} -- The device code returned by the verification code method

\end{description}\end{quote}

\end{fulllineitems}

\index{setAppSessionState() (BaseSpacePy.api.BaseSpaceAPI.BaseSpaceAPI method)}

\begin{fulllineitems}
\phantomsection\label{Available modules:BaseSpacePy.api.BaseSpaceAPI.BaseSpaceAPI.setAppSessionState}\pysiglinewithargsret{\bfcode{setAppSessionState}}{\emph{Id}, \emph{Status}, \emph{Summary}}{}
Set the status of an AppResult object
\begin{quote}\begin{description}
\item[{Parameters}] \leavevmode\begin{itemize}
\item {} 
\textbf{Id} -- The id of the AppResult

\item {} 
\textbf{Status} -- The status assignment string must

\item {} 
\textbf{Summary} -- The summary string

\end{itemize}

\end{description}\end{quote}

\end{fulllineitems}

\index{setTimeout() (BaseSpacePy.api.BaseSpaceAPI.BaseSpaceAPI method)}

\begin{fulllineitems}
\phantomsection\label{Available modules:BaseSpacePy.api.BaseSpaceAPI.BaseSpaceAPI.setTimeout}\pysiglinewithargsret{\bfcode{setTimeout}}{\emph{time}}{}
Specify the timeout in seconds for each request made
\begin{quote}\begin{description}
\item[{Parameters}] \leavevmode
\textbf{time} -- timeout in second

\end{description}\end{quote}

\end{fulllineitems}


\end{fulllineitems}



\section{Models}
\label{Available modules:models}

\subsection{Project}
\label{Available modules:project}\index{Project (class in BaseSpacePy.model.Project)}

\begin{fulllineitems}
\phantomsection\label{Available modules:BaseSpacePy.model.Project.Project}\pysigline{\strong{class }\code{BaseSpacePy.model.Project.}\bfcode{Project}}
Represents a BaseSpace Project object.
\index{createAppResult() (BaseSpacePy.model.Project.Project method)}

\begin{fulllineitems}
\phantomsection\label{Available modules:BaseSpacePy.model.Project.Project.createAppResult}\pysiglinewithargsret{\bfcode{createAppResult}}{\emph{api}, \emph{name}, \emph{desc}, \emph{appSessionId=None}, \emph{samples=}\optional{}}{}
Return a newly created app result object
\begin{quote}\begin{description}
\item[{Parameters}] \leavevmode\begin{itemize}
\item {} 
\textbf{api} -- An instance of BaseSpaceAPI

\item {} 
\textbf{name} -- The name of the app result

\item {} 
\textbf{desc} -- A describtion of the app result

\end{itemize}

\end{description}\end{quote}

\end{fulllineitems}

\index{getAccessStr() (BaseSpacePy.model.Project.Project method)}

\begin{fulllineitems}
\phantomsection\label{Available modules:BaseSpacePy.model.Project.Project.getAccessStr}\pysiglinewithargsret{\bfcode{getAccessStr}}{\emph{scope='write'}}{}
Returns the scope-string to used for requesting BaseSpace access to the object
\begin{quote}\begin{description}
\item[{Parameters}] \leavevmode
\textbf{scope} -- The scope-type that is request (write\textbar{}read)

\end{description}\end{quote}

\end{fulllineitems}

\index{getAppResults() (BaseSpacePy.model.Project.Project method)}

\begin{fulllineitems}
\phantomsection\label{Available modules:BaseSpacePy.model.Project.Project.getAppResults}\pysiglinewithargsret{\bfcode{getAppResults}}{\emph{api}, \emph{myQp=\{\}}, \emph{statuses=}\optional{}}{}
Returns a list of AppResult objects.
\begin{quote}\begin{description}
\item[{Parameters}] \leavevmode\begin{itemize}
\item {} 
\textbf{api} -- An instance of BaseSpaceAPI

\item {} 
\textbf{statuses} -- An optional list of statuses

\end{itemize}

\end{description}\end{quote}

\end{fulllineitems}

\index{getSamples() (BaseSpacePy.model.Project.Project method)}

\begin{fulllineitems}
\phantomsection\label{Available modules:BaseSpacePy.model.Project.Project.getSamples}\pysiglinewithargsret{\bfcode{getSamples}}{\emph{api}}{}
Returns a list of Sample objects.
\begin{quote}\begin{description}
\item[{Parameters}] \leavevmode
\textbf{api} -- An instance of BaseSpaceAPI

\end{description}\end{quote}

\end{fulllineitems}

\index{isInit() (BaseSpacePy.model.Project.Project method)}

\begin{fulllineitems}
\phantomsection\label{Available modules:BaseSpacePy.model.Project.Project.isInit}\pysiglinewithargsret{\bfcode{isInit}}{}{}
Is called to test if the Project instance has been initialized.
\begin{description}
\item[{Throws:}] \leavevmode
ModelNotInitializedException - Indicates the object has not been populated yet.

\end{description}

\end{fulllineitems}


\end{fulllineitems}



\subsection{AppSession and AppResult}
\label{Available modules:appsession-and-appresult}\index{AppSession (class in BaseSpacePy.model.AppSession)}

\begin{fulllineitems}
\phantomsection\label{Available modules:BaseSpacePy.model.AppSession.AppSession}\pysigline{\strong{class }\code{BaseSpacePy.model.AppSession.}\bfcode{AppSession}}
AppLaunch contains the data returned
\index{setStatus() (BaseSpacePy.model.AppSession.AppSession method)}

\begin{fulllineitems}
\phantomsection\label{Available modules:BaseSpacePy.model.AppSession.AppSession.setStatus}\pysiglinewithargsret{\bfcode{setStatus}}{\emph{api}, \emph{Status}, \emph{Summary}}{}
Sets the status of the AppSession (note: once set to `completed' or `aborted' no more work can be done to the instance)
\begin{quote}\begin{description}
\item[{Parameters}] \leavevmode\begin{itemize}
\item {} 
\textbf{api} -- An instance of BaseSpaceAPI

\item {} 
\textbf{Status} -- The status value, must be completed, aborted, working, or suspended

\item {} 
\textbf{Summary} -- The status summary

\end{itemize}

\end{description}\end{quote}

\end{fulllineitems}


\end{fulllineitems}

\index{AppResult (class in BaseSpacePy.model.AppResult)}

\begin{fulllineitems}
\phantomsection\label{Available modules:BaseSpacePy.model.AppResult.AppResult}\pysigline{\strong{class }\code{BaseSpacePy.model.AppResult.}\bfcode{AppResult}}~\index{getAccessStr() (BaseSpacePy.model.AppResult.AppResult method)}

\begin{fulllineitems}
\phantomsection\label{Available modules:BaseSpacePy.model.AppResult.AppResult.getAccessStr}\pysiglinewithargsret{\bfcode{getAccessStr}}{\emph{scope='write'}}{}
Returns the scope-string to be used for requesting BaseSpace access to the object
\begin{quote}\begin{description}
\item[{Parameters}] \leavevmode
\textbf{scope} -- The scope-type that is request (write\textbar{}read)

\end{description}\end{quote}

\end{fulllineitems}

\index{getFiles() (BaseSpacePy.model.AppResult.AppResult method)}

\begin{fulllineitems}
\phantomsection\label{Available modules:BaseSpacePy.model.AppResult.AppResult.getFiles}\pysiglinewithargsret{\bfcode{getFiles}}{\emph{api}, \emph{myQp=\{\}}}{}
Returns a list of file objects
\begin{quote}\begin{description}
\item[{Parameters}] \leavevmode\begin{itemize}
\item {} 
\textbf{api} -- An instance of BaseSpaceAPI

\item {} 
\textbf{myQp} -- (Optional) QueryParameters for sorting and filtering the file list

\end{itemize}

\end{description}\end{quote}

\end{fulllineitems}

\index{getReferencedSamples() (BaseSpacePy.model.AppResult.AppResult method)}

\begin{fulllineitems}
\phantomsection\label{Available modules:BaseSpacePy.model.AppResult.AppResult.getReferencedSamples}\pysiglinewithargsret{\bfcode{getReferencedSamples}}{\emph{api}}{}
Returns a list of sample objects references by the AppResult. NOTE this method makes one request to REST server per sample

\end{fulllineitems}

\index{getReferencedSamplesIds() (BaseSpacePy.model.AppResult.AppResult method)}

\begin{fulllineitems}
\phantomsection\label{Available modules:BaseSpacePy.model.AppResult.AppResult.getReferencedSamplesIds}\pysiglinewithargsret{\bfcode{getReferencedSamplesIds}}{}{}
Return a list of sample ids for the samples referenced.

\end{fulllineitems}

\index{isInit() (BaseSpacePy.model.AppResult.AppResult method)}

\begin{fulllineitems}
\phantomsection\label{Available modules:BaseSpacePy.model.AppResult.AppResult.isInit}\pysiglinewithargsret{\bfcode{isInit}}{}{}
Is called to test if the Project instance has been initialized
\begin{description}
\item[{Throws:}] \leavevmode
ModelNotInitializedException  - if the instance has not been populated.

\end{description}

\end{fulllineitems}

\index{uploadFile() (BaseSpacePy.model.AppResult.AppResult method)}

\begin{fulllineitems}
\phantomsection\label{Available modules:BaseSpacePy.model.AppResult.AppResult.uploadFile}\pysiglinewithargsret{\bfcode{uploadFile}}{\emph{api}, \emph{localPath}, \emph{fileName}, \emph{directory}, \emph{contentType}}{}
Uploads a local file to the BaseSpace AppResult
\begin{quote}\begin{description}
\item[{Parameters}] \leavevmode\begin{itemize}
\item {} 
\textbf{api} -- An instance of BaseSpaceAPI

\item {} 
\textbf{localPath} -- The local path of the file

\item {} 
\textbf{fileName} -- The filename

\item {} 
\textbf{directory} -- The remote directory to upload to

\item {} 
\textbf{contentType} -- The content-type of the file

\end{itemize}

\end{description}\end{quote}

\end{fulllineitems}


\end{fulllineitems}



\subsection{Sample}
\label{Available modules:sample}\index{Sample (class in BaseSpacePy.model.Sample)}

\begin{fulllineitems}
\phantomsection\label{Available modules:BaseSpacePy.model.Sample.Sample}\pysigline{\strong{class }\code{BaseSpacePy.model.Sample.}\bfcode{Sample}}
Representation of a BaseSpace Sample object.
\index{getAccessStr() (BaseSpacePy.model.Sample.Sample method)}

\begin{fulllineitems}
\phantomsection\label{Available modules:BaseSpacePy.model.Sample.Sample.getAccessStr}\pysiglinewithargsret{\bfcode{getAccessStr}}{\emph{scope='write'}}{}
Returns the scope-string to used for requesting BaseSpace access to the sample.
\begin{quote}\begin{description}
\item[{Parameters}] \leavevmode
\textbf{scope} -- The scope type that is request (write\textbar{}read).

\end{description}\end{quote}

\end{fulllineitems}

\index{getFiles() (BaseSpacePy.model.Sample.Sample method)}

\begin{fulllineitems}
\phantomsection\label{Available modules:BaseSpacePy.model.Sample.Sample.getFiles}\pysiglinewithargsret{\bfcode{getFiles}}{\emph{api}, \emph{myQp=\{\}}}{}
Returns a list of File objects
\begin{quote}\begin{description}
\item[{Parameters}] \leavevmode\begin{itemize}
\item {} 
\textbf{api} -- A BaseSpaceAPI instance

\item {} 
\textbf{myQp} -- Query parameters to sort and filter the file list by.

\end{itemize}

\end{description}\end{quote}

\end{fulllineitems}

\index{getReferencedAppResults() (BaseSpacePy.model.Sample.Sample method)}

\begin{fulllineitems}
\phantomsection\label{Available modules:BaseSpacePy.model.Sample.Sample.getReferencedAppResults}\pysiglinewithargsret{\bfcode{getReferencedAppResults}}{\emph{api}}{}
Return the AppResults referenced by this sample. Note the returned AppResult objects
do not have their ``References'' field set, to get a fully populate AppResult object
you must use getAppResultById in BaseSpaceAPI.

\end{fulllineitems}

\index{isInit() (BaseSpacePy.model.Sample.Sample method)}

\begin{fulllineitems}
\phantomsection\label{Available modules:BaseSpacePy.model.Sample.Sample.isInit}\pysiglinewithargsret{\bfcode{isInit}}{}{}
Is called to test if the sample instance has been initialized.
\begin{description}
\item[{Throws:}] \leavevmode
ModelNotInitializedException - Indicated the Id variable is not set.

\end{description}

\end{fulllineitems}


\end{fulllineitems}



\subsection{File}
\label{Available modules:file}\index{File (class in BaseSpacePy.model.File)}

\begin{fulllineitems}
\phantomsection\label{Available modules:BaseSpacePy.model.File.File}\pysigline{\strong{class }\code{BaseSpacePy.model.File.}\bfcode{File}}
Represents a BaseSpace file object.
\index{downloadFile() (BaseSpacePy.model.File.File method)}

\begin{fulllineitems}
\phantomsection\label{Available modules:BaseSpacePy.model.File.File.downloadFile}\pysiglinewithargsret{\bfcode{downloadFile}}{\emph{api}, \emph{localDir}, \emph{range=}\optional{}}{}
Download the file object to the specified localDir or a byte range of the file, by specifying the 
start and stop byte in the range.
\begin{quote}\begin{description}
\item[{Parameters}] \leavevmode\begin{itemize}
\item {} 
\textbf{api} -- A BaseSpaceAPI with read access on the scope including the file object.

\item {} 
\textbf{loadlDir} -- The local directory to place the file in.

\item {} 
\textbf{range} -- Specify the start and stop byte of the file chunk that needs retrieved.

\end{itemize}

\end{description}\end{quote}

\end{fulllineitems}

\index{filterVariant() (BaseSpacePy.model.File.File method)}

\begin{fulllineitems}
\phantomsection\label{Available modules:BaseSpacePy.model.File.File.filterVariant}\pysiglinewithargsret{\bfcode{filterVariant}}{\emph{api}, \emph{Chrom}, \emph{StartPos}, \emph{EndPos}, \emph{q=None}}{}
Returns a list of Variant objects available in the specified region
\begin{quote}\begin{description}
\item[{Parameters}] \leavevmode\begin{itemize}
\item {} 
\textbf{api} -- An instance of BaseSpaceAPI

\item {} 
\textbf{Chrom} -- Chromosome as a string - for example `chr2'

\item {} 
\textbf{StartPos} -- The start position of region of interest as a string

\item {} 
\textbf{EndPos} -- The end position of region of interest as a string

\item {} 
\textbf{q} -- An instance of

\end{itemize}

\end{description}\end{quote}

\end{fulllineitems}

\index{getCoverageMeta() (BaseSpacePy.model.File.File method)}

\begin{fulllineitems}
\phantomsection\label{Available modules:BaseSpacePy.model.File.File.getCoverageMeta}\pysiglinewithargsret{\bfcode{getCoverageMeta}}{\emph{api}, \emph{Chrom}}{}
Return an object of CoverageMetadata for the selected region
\begin{quote}\begin{description}
\item[{Parameters}] \leavevmode\begin{itemize}
\item {} 
\textbf{api} -- An instance of BaseSpaceAPI.

\item {} 
\textbf{Chrom} -- The chromosome of interest.

\end{itemize}

\end{description}\end{quote}

\end{fulllineitems}

\index{getIntervalCoverage() (BaseSpacePy.model.File.File method)}

\begin{fulllineitems}
\phantomsection\label{Available modules:BaseSpacePy.model.File.File.getIntervalCoverage}\pysiglinewithargsret{\bfcode{getIntervalCoverage}}{\emph{api}, \emph{Chrom}, \emph{StartPos}, \emph{EndPos}}{}
Return a coverage object for the specified region and chromosome.
\begin{quote}\begin{description}
\item[{Parameters}] \leavevmode\begin{itemize}
\item {} 
\textbf{api} -- An instance of BaseSpaceAPI

\item {} 
\textbf{Chrom} -- Chromosome as a string - for example `chr2'

\item {} 
\textbf{StartPos} -- The start position of region of interest as a string

\item {} 
\textbf{EndPos} -- The end position of region of interest as a string

\end{itemize}

\end{description}\end{quote}

\end{fulllineitems}

\index{getVariantMeta() (BaseSpacePy.model.File.File method)}

\begin{fulllineitems}
\phantomsection\label{Available modules:BaseSpacePy.model.File.File.getVariantMeta}\pysiglinewithargsret{\bfcode{getVariantMeta}}{\emph{api}}{}
Return the the meta info for a VCF file as a VariantInfo object
\begin{quote}\begin{description}
\item[{Parameters}] \leavevmode
\textbf{api} -- An instance of BaseSpaceAPI

\end{description}\end{quote}

\end{fulllineitems}

\index{isInit() (BaseSpacePy.model.File.File method)}

\begin{fulllineitems}
\phantomsection\label{Available modules:BaseSpacePy.model.File.File.isInit}\pysiglinewithargsret{\bfcode{isInit}}{}{}
Is called to test if the File instance has been initialized.
\begin{description}
\item[{Throws:}] \leavevmode
ModelNotInitializedException if the instance has not been populated yet.

\end{description}

\end{fulllineitems}

\index{isValidFileOption() (BaseSpacePy.model.File.File method)}

\begin{fulllineitems}
\phantomsection\label{Available modules:BaseSpacePy.model.File.File.isValidFileOption}\pysiglinewithargsret{\bfcode{isValidFileOption}}{\emph{filetype}}{}
Is called to test if the File instance is matches the filtype parameter
\begin{quote}\begin{description}
\item[{Parameters}] \leavevmode
\textbf{filetype} -- The filetype for coverage or variant requests

\end{description}\end{quote}

\end{fulllineitems}


\end{fulllineitems}



\subsection{QueryParameters}
\label{Available modules:queryparameters}\index{QueryParameters (class in BaseSpacePy.model.QueryParameters)}

\begin{fulllineitems}
\phantomsection\label{Available modules:BaseSpacePy.model.QueryParameters.QueryParameters}\pysiglinewithargsret{\strong{class }\code{BaseSpacePy.model.QueryParameters.}\bfcode{QueryParameters}}{\emph{pars=\{\}, required={[}'SortBy', `Offset', `Limit', `SortDir'{]}}}{}
The QueryParameters class can be passed as an optional arguments for a specific sorting of list-responses (such as lists of sample, AppResult, or variants)

\end{fulllineitems}



\chapter{Indices and tables}
\label{index:indices-and-tables}\begin{itemize}
\item {} 
\emph{genindex}

\item {} 
\emph{modindex}

\item {} 
\emph{search}

\end{itemize}



\renewcommand{\indexname}{Index}
\printindex
\end{document}
