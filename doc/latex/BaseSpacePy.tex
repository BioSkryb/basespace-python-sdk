% Generated by Sphinx.
\def\sphinxdocclass{report}
\documentclass[letterpaper,10pt,english]{sphinxmanual}
\usepackage[utf8]{inputenc}
\DeclareUnicodeCharacter{00A0}{\nobreakspace}
\usepackage[T1]{fontenc}
\usepackage{babel}
\usepackage{times}
\usepackage[Bjarne]{fncychap}
\usepackage{longtable}
\usepackage{sphinx}
\usepackage{multirow}


\title{BaseSpacePy Documentation}
\date{August 15, 2012}
\release{0.1}
\author{Morten Kallberg}
\newcommand{\sphinxlogo}{}
\renewcommand{\releasename}{Release}
\makeindex

\makeatletter
\def\PYG@reset{\let\PYG@it=\relax \let\PYG@bf=\relax%
    \let\PYG@ul=\relax \let\PYG@tc=\relax%
    \let\PYG@bc=\relax \let\PYG@ff=\relax}
\def\PYG@tok#1{\csname PYG@tok@#1\endcsname}
\def\PYG@toks#1+{\ifx\relax#1\empty\else%
    \PYG@tok{#1}\expandafter\PYG@toks\fi}
\def\PYG@do#1{\PYG@bc{\PYG@tc{\PYG@ul{%
    \PYG@it{\PYG@bf{\PYG@ff{#1}}}}}}}
\def\PYG#1#2{\PYG@reset\PYG@toks#1+\relax+\PYG@do{#2}}

\def\PYG@tok@gd{\def\PYG@tc##1{\textcolor[rgb]{0.63,0.00,0.00}{##1}}}
\def\PYG@tok@gu{\let\PYG@bf=\textbf\def\PYG@tc##1{\textcolor[rgb]{0.50,0.00,0.50}{##1}}}
\def\PYG@tok@gt{\def\PYG@tc##1{\textcolor[rgb]{0.00,0.25,0.82}{##1}}}
\def\PYG@tok@gs{\let\PYG@bf=\textbf}
\def\PYG@tok@gr{\def\PYG@tc##1{\textcolor[rgb]{1.00,0.00,0.00}{##1}}}
\def\PYG@tok@cm{\let\PYG@it=\textit\def\PYG@tc##1{\textcolor[rgb]{0.25,0.50,0.56}{##1}}}
\def\PYG@tok@vg{\def\PYG@tc##1{\textcolor[rgb]{0.73,0.38,0.84}{##1}}}
\def\PYG@tok@m{\def\PYG@tc##1{\textcolor[rgb]{0.13,0.50,0.31}{##1}}}
\def\PYG@tok@mh{\def\PYG@tc##1{\textcolor[rgb]{0.13,0.50,0.31}{##1}}}
\def\PYG@tok@cs{\def\PYG@tc##1{\textcolor[rgb]{0.25,0.50,0.56}{##1}}\def\PYG@bc##1{\colorbox[rgb]{1.00,0.94,0.94}{##1}}}
\def\PYG@tok@ge{\let\PYG@it=\textit}
\def\PYG@tok@vc{\def\PYG@tc##1{\textcolor[rgb]{0.73,0.38,0.84}{##1}}}
\def\PYG@tok@il{\def\PYG@tc##1{\textcolor[rgb]{0.13,0.50,0.31}{##1}}}
\def\PYG@tok@go{\def\PYG@tc##1{\textcolor[rgb]{0.19,0.19,0.19}{##1}}}
\def\PYG@tok@cp{\def\PYG@tc##1{\textcolor[rgb]{0.00,0.44,0.13}{##1}}}
\def\PYG@tok@gi{\def\PYG@tc##1{\textcolor[rgb]{0.00,0.63,0.00}{##1}}}
\def\PYG@tok@gh{\let\PYG@bf=\textbf\def\PYG@tc##1{\textcolor[rgb]{0.00,0.00,0.50}{##1}}}
\def\PYG@tok@ni{\let\PYG@bf=\textbf\def\PYG@tc##1{\textcolor[rgb]{0.84,0.33,0.22}{##1}}}
\def\PYG@tok@nl{\let\PYG@bf=\textbf\def\PYG@tc##1{\textcolor[rgb]{0.00,0.13,0.44}{##1}}}
\def\PYG@tok@nn{\let\PYG@bf=\textbf\def\PYG@tc##1{\textcolor[rgb]{0.05,0.52,0.71}{##1}}}
\def\PYG@tok@no{\def\PYG@tc##1{\textcolor[rgb]{0.38,0.68,0.84}{##1}}}
\def\PYG@tok@na{\def\PYG@tc##1{\textcolor[rgb]{0.25,0.44,0.63}{##1}}}
\def\PYG@tok@nb{\def\PYG@tc##1{\textcolor[rgb]{0.00,0.44,0.13}{##1}}}
\def\PYG@tok@nc{\let\PYG@bf=\textbf\def\PYG@tc##1{\textcolor[rgb]{0.05,0.52,0.71}{##1}}}
\def\PYG@tok@nd{\let\PYG@bf=\textbf\def\PYG@tc##1{\textcolor[rgb]{0.33,0.33,0.33}{##1}}}
\def\PYG@tok@ne{\def\PYG@tc##1{\textcolor[rgb]{0.00,0.44,0.13}{##1}}}
\def\PYG@tok@nf{\def\PYG@tc##1{\textcolor[rgb]{0.02,0.16,0.49}{##1}}}
\def\PYG@tok@si{\let\PYG@it=\textit\def\PYG@tc##1{\textcolor[rgb]{0.44,0.63,0.82}{##1}}}
\def\PYG@tok@s2{\def\PYG@tc##1{\textcolor[rgb]{0.25,0.44,0.63}{##1}}}
\def\PYG@tok@vi{\def\PYG@tc##1{\textcolor[rgb]{0.73,0.38,0.84}{##1}}}
\def\PYG@tok@nt{\let\PYG@bf=\textbf\def\PYG@tc##1{\textcolor[rgb]{0.02,0.16,0.45}{##1}}}
\def\PYG@tok@nv{\def\PYG@tc##1{\textcolor[rgb]{0.73,0.38,0.84}{##1}}}
\def\PYG@tok@s1{\def\PYG@tc##1{\textcolor[rgb]{0.25,0.44,0.63}{##1}}}
\def\PYG@tok@gp{\let\PYG@bf=\textbf\def\PYG@tc##1{\textcolor[rgb]{0.78,0.36,0.04}{##1}}}
\def\PYG@tok@sh{\def\PYG@tc##1{\textcolor[rgb]{0.25,0.44,0.63}{##1}}}
\def\PYG@tok@ow{\let\PYG@bf=\textbf\def\PYG@tc##1{\textcolor[rgb]{0.00,0.44,0.13}{##1}}}
\def\PYG@tok@sx{\def\PYG@tc##1{\textcolor[rgb]{0.78,0.36,0.04}{##1}}}
\def\PYG@tok@bp{\def\PYG@tc##1{\textcolor[rgb]{0.00,0.44,0.13}{##1}}}
\def\PYG@tok@c1{\let\PYG@it=\textit\def\PYG@tc##1{\textcolor[rgb]{0.25,0.50,0.56}{##1}}}
\def\PYG@tok@kc{\let\PYG@bf=\textbf\def\PYG@tc##1{\textcolor[rgb]{0.00,0.44,0.13}{##1}}}
\def\PYG@tok@c{\let\PYG@it=\textit\def\PYG@tc##1{\textcolor[rgb]{0.25,0.50,0.56}{##1}}}
\def\PYG@tok@mf{\def\PYG@tc##1{\textcolor[rgb]{0.13,0.50,0.31}{##1}}}
\def\PYG@tok@err{\def\PYG@bc##1{\fcolorbox[rgb]{1.00,0.00,0.00}{1,1,1}{##1}}}
\def\PYG@tok@kd{\let\PYG@bf=\textbf\def\PYG@tc##1{\textcolor[rgb]{0.00,0.44,0.13}{##1}}}
\def\PYG@tok@ss{\def\PYG@tc##1{\textcolor[rgb]{0.32,0.47,0.09}{##1}}}
\def\PYG@tok@sr{\def\PYG@tc##1{\textcolor[rgb]{0.14,0.33,0.53}{##1}}}
\def\PYG@tok@mo{\def\PYG@tc##1{\textcolor[rgb]{0.13,0.50,0.31}{##1}}}
\def\PYG@tok@mi{\def\PYG@tc##1{\textcolor[rgb]{0.13,0.50,0.31}{##1}}}
\def\PYG@tok@kn{\let\PYG@bf=\textbf\def\PYG@tc##1{\textcolor[rgb]{0.00,0.44,0.13}{##1}}}
\def\PYG@tok@o{\def\PYG@tc##1{\textcolor[rgb]{0.40,0.40,0.40}{##1}}}
\def\PYG@tok@kr{\let\PYG@bf=\textbf\def\PYG@tc##1{\textcolor[rgb]{0.00,0.44,0.13}{##1}}}
\def\PYG@tok@s{\def\PYG@tc##1{\textcolor[rgb]{0.25,0.44,0.63}{##1}}}
\def\PYG@tok@kp{\def\PYG@tc##1{\textcolor[rgb]{0.00,0.44,0.13}{##1}}}
\def\PYG@tok@w{\def\PYG@tc##1{\textcolor[rgb]{0.73,0.73,0.73}{##1}}}
\def\PYG@tok@kt{\def\PYG@tc##1{\textcolor[rgb]{0.56,0.13,0.00}{##1}}}
\def\PYG@tok@sc{\def\PYG@tc##1{\textcolor[rgb]{0.25,0.44,0.63}{##1}}}
\def\PYG@tok@sb{\def\PYG@tc##1{\textcolor[rgb]{0.25,0.44,0.63}{##1}}}
\def\PYG@tok@k{\let\PYG@bf=\textbf\def\PYG@tc##1{\textcolor[rgb]{0.00,0.44,0.13}{##1}}}
\def\PYG@tok@se{\let\PYG@bf=\textbf\def\PYG@tc##1{\textcolor[rgb]{0.25,0.44,0.63}{##1}}}
\def\PYG@tok@sd{\let\PYG@it=\textit\def\PYG@tc##1{\textcolor[rgb]{0.25,0.44,0.63}{##1}}}

\def\PYGZbs{\char`\\}
\def\PYGZus{\char`\_}
\def\PYGZob{\char`\{}
\def\PYGZcb{\char`\}}
\def\PYGZca{\char`\^}
\def\PYGZsh{\char`\#}
\def\PYGZpc{\char`\%}
\def\PYGZdl{\char`\$}
\def\PYGZti{\char`\~}
% for compatibility with earlier versions
\def\PYGZat{@}
\def\PYGZlb{[}
\def\PYGZrb{]}
\makeatother

\begin{document}

\maketitle
\tableofcontents
\phantomsection\label{index::doc}



\chapter{Getting Started}
\label{Getting Started:getting-started}\label{Getting Started::doc}\label{Getting Started:basespacepy}

\section{Introduction}
\label{Getting Started:introduction}
\code{BaseSpacePy} is a Python based SDK to be used in the development of Apps and scripts for working with Illumina's BaseSpace cloud-computing solution for next-gen sequencing data analysis.
The primary purpose of the SDK is to provide an easy-to-use Python environment enabling developers to authenticate a user, retrieve data, and upload data/results from their own analysis to BaseSpace.

If you haven't already done so, you may wish to familiarize yourself with the BaseSpace developer documentation (\href{https://developer.basespace.illumina.com/}{https://developer.basespace.illumina.com/}) prior to working through the example scripts below.

\begin{notice}{note}{Note:}
It will be necessary to have created a BaseSpace account with a new App and have the \code{client\_key} and \code{client\_secret} codes for the App available to run a number of the following examples.
\end{notice}


\subsection{Availability}
\label{Getting Started:availability}
Version 0.1 of \code{BaseSpacePy} can be checked out here:

\begin{Verbatim}[commandchars=\\\{\}]
\PYG{n}{svn} \PYG{n}{checkout} \PYG{n}{svn}\PYG{p}{:}\PYG{o}{/}\PYG{o}{/}\PYG{n}{ilmnhw}\PYG{o}{-}\PYG{n}{biolinsd}\PYG{o}{.}\PYG{n}{hwus}\PYG{o}{.}\PYG{n}{illumina}\PYG{o}{.}\PYG{n}{com}\PYG{p}{:}\PYG{o}{/}\PYG{n}{data}\PYG{o}{/}\PYG{n}{tmann}\PYG{o}{/}\PYG{n}{svn}\PYG{o}{/}\PYG{n}{svnrepos}\PYG{o}{/}\PYG{n}{BaseSpacePy\PYGZus{}v0}\PYG{o}{.}\PYG{l+m+mi}{1}
\end{Verbatim}


\subsection{Setup}
\label{Getting Started:setup}
\emph{Requirements:} Python 2.6 with the packages `urllib2', `pycurl', and `shutil' available.

To install `BaseSpacePy' run the `setup.py' script in the \code{src} directory (for a global install you will need to run this command with root privileges):

\begin{Verbatim}[commandchars=\\\{\}]
\PYG{n}{cd} \PYG{n}{BaseSpacePy\PYGZus{}v0}\PYG{o}{.}\PYG{l+m+mi}{1}\PYG{o}{/}\PYG{n}{src}
\PYG{n}{python} \PYG{n}{setup}\PYG{o}{.}\PYG{n}{py} \PYG{n}{install}
\end{Verbatim}

If you do not have root access, you may use the \code{-{-}prefix} option to specify the install directory (make sure this directory is in you PYTHONPATH):

\begin{Verbatim}[commandchars=\\\{\}]
\PYG{n}{python} \PYG{n}{setup}\PYG{o}{.}\PYG{n}{py} \PYG{n}{install} \PYG{o}{-}\PYG{o}{-}\PYG{n}{prefix}\PYG{o}{=}\PYG{o}{/}\PYG{n}{folder}\PYG{o}{/}\PYG{o+ow}{in}\PYG{o}{/}\PYG{n}{my}\PYG{o}{/}\PYG{n}{pythonpath}
\end{Verbatim}

For more install options type:

\begin{Verbatim}[commandchars=\\\{\}]
\PYG{n}{python} \PYG{n}{setup}\PYG{o}{.}\PYG{n}{py} \PYG{o}{-}\PYG{o}{-}\PYG{n}{help}
\end{Verbatim}

Altenatively you may include the src directory in your PYTHONPATH by doing the following export:

\begin{Verbatim}[commandchars=\\\{\}]
export PYTHONPATH=\$PYTHONPATH:/my/path/BaseSpacePy\_vx.x/src
\end{Verbatim}

or add it to the PYTHONPATH at the top of your Python scripts using BaseSpacePy:

\begin{Verbatim}[commandchars=\\\{\}]
\PYG{k+kn}{import} \PYG{n+nn}{sys}
\PYG{n}{sys}\PYG{o}{.}\PYG{n}{path}\PYG{o}{.}\PYG{n}{append}\PYG{p}{(}\PYG{l+s}{'}\PYG{l+s}{/my/path/BaseSpacePy\PYGZus{}vx.x/src}\PYG{l+s}{'}\PYG{p}{)}
\PYG{k+kn}{import} \PYG{n+nn}{BaseSpacePy}
\end{Verbatim}

To test that everything is working as expected, launch a Python prompt and try importing `BaseSpacePy':

\begin{Verbatim}[commandchars=\\\{\}]
mkallberg@ubuntu:\textasciitilde{}/\$ python
\textgreater{}\textgreater{}\textgreater{} import BaseSpacePy
\end{Verbatim}


\section{Application triggering}
\label{Getting Started:application-triggering}
This section demonstrates how to retrieve the \code{AppLaunch} object produced when a user triggers a BaseSpace App.
Further, we cover how to automatically generate the scope strings to request access to the data object (be it a project, a sample, or an analysis)
that the App was triggered to analyze.

The initial http request to our App from BaseSpace is identified by an \code{ApplicationActionId}, using this piece of information
we are able to obtain information about the user who launched the App and the data that is sought analyzed by the App.
First, we instantiate a BaseSpaceAuth object using the \code{client\_key} and \code{client\_secret} codes provided on the BaseSpace developers website when registering our App:

\begin{Verbatim}[commandchars=\\\{\}]
\PYG{k+kn}{from} \PYG{n+nn}{BaseSpacePy.api.BaseSpaceAuth} \PYG{k+kn}{import} \PYG{n}{BaseSpaceAuth}

\PYG{c}{\PYGZsh{} initialize an authentication object using the key and secret from your app}
\PYG{c}{\PYGZsh{} Fill in with your own values}
\PYG{n}{client\PYGZus{}key}                 \PYG{o}{=} \PYG{o}{\textless{}}\PYG{n}{my} \PYG{n}{key}\PYG{o}{\textgreater{}}
\PYG{n}{client\PYGZus{}secret}              \PYG{o}{=} \PYG{o}{\textless{}}\PYG{n}{my} \PYG{n}{secret}\PYG{o}{\textgreater{}}
\PYG{n}{ApplicationActionId}        \PYG{o}{=} \PYG{o}{\textless{}}\PYG{n}{my} \PYG{n}{action} \PYG{n+nb}{id}\PYG{o}{\textgreater{}}
\PYG{n}{BaseSpaceUrl}               \PYG{o}{=} \PYG{l+s}{'}\PYG{l+s}{https://api.cloud-endor.illumina.com/}\PYG{l+s}{'}
\PYG{n}{version}                    \PYG{o}{=} \PYG{l+s}{'}\PYG{l+s}{v1pre2/}\PYG{l+s}{'}

\PYG{c}{\PYGZsh{} First we will initialize a BaseSpace authentication object}
\PYG{n}{BSauth} \PYG{o}{=} \PYG{n}{BaseSpaceAuth}\PYG{p}{(}\PYG{n}{client\PYGZus{}key}\PYG{p}{,}\PYG{n}{client\PYGZus{}secret}\PYG{p}{,}\PYG{n}{BaseSpaceUrl}\PYG{p}{,}\PYG{n}{version}\PYG{p}{)}

\PYG{c}{\PYGZsh{} By supplying the application trigger id we can get out an AppLaunch object}
\PYG{n}{triggerObj} \PYG{o}{=} \PYG{n}{BSauth}\PYG{o}{.}\PYG{n}{getAppTrigger}\PYG{p}{(}\PYG{n}{ApplicationActionId}\PYG{p}{)}
\PYG{k}{print} \PYG{n+nb}{str}\PYG{p}{(}\PYG{n}{triggerObj}\PYG{p}{)}
\end{Verbatim}

\begin{Verbatim}[commandchars=\\\{\}]
\PYG{n}{Output}\PYG{p}{[}\PYG{p}{]}\PYG{p}{:}

\PYG{n}{https}\PYG{p}{:}\PYG{o}{/}\PYG{o}{/}\PYG{n}{api}\PYG{o}{.}\PYG{n}{cloud}\PYG{o}{-}\PYG{n}{endor}\PYG{o}{.}\PYG{n}{illumina}\PYG{o}{.}\PYG{n}{com}\PYG{o}{/}\PYG{n}{v1pre2}\PYG{o}{/}\PYG{n}{applicationactions}\PYG{o}{/}\PYG{o}{\textless{}}\PYG{n}{my} \PYG{n}{action} \PYG{n+nb}{id}\PYG{o}{\textgreater{}}
\end{Verbatim}

We can get the type of object the app was triggered on from the \code{getLaunchType}-method in the \code{BaseSpaceAuth} instance:

\begin{Verbatim}[commandchars=\\\{\}]
\PYG{c}{\PYGZsh{} The trigger type is a list with two items, the first a string taking the one of the values ('Projects','Samples','Analyses')}
\PYG{c}{\PYGZsh{} and the second a list of the objects of that type}
\PYG{n}{triggerType} \PYG{o}{=} \PYG{n}{triggerObj}\PYG{o}{.}\PYG{n}{getLaunchType}\PYG{p}{(}\PYG{p}{)}
\PYG{k}{print} \PYG{l+s}{"}\PYG{l+s+se}{\PYGZbs{}n}\PYG{l+s}{Type of data the app was triggered on}\PYG{l+s}{"}
\PYG{k}{print} \PYG{n}{triggerType}
\PYG{k}{print} \PYG{l+s}{"}\PYG{l+s+se}{\PYGZbs{}n}\PYG{l+s}{We can get a handle for the user who triggered the app}\PYG{l+s+se}{\PYGZbs{}n}\PYG{l+s}{"} \PYG{o}{+} \PYG{n+nb}{str}\PYG{p}{(}\PYG{n}{triggerObj}\PYG{o}{.}\PYG{n}{User}\PYG{p}{)}
\end{Verbatim}

\begin{Verbatim}[commandchars=\\\{\}]
\PYG{n}{Output}\PYG{p}{[}\PYG{p}{]}\PYG{p}{:}

\PYG{n}{Type} \PYG{n}{of} \PYG{n}{data} \PYG{n}{the} \PYG{n}{app} \PYG{n}{was} \PYG{n}{triggered} \PYG{n}{on}
\PYG{p}{[}\PYG{l+s}{'}\PYG{l+s}{Projects}\PYG{l+s}{'}\PYG{p}{,} \PYG{p}{[}\PYG{n}{YourProject}\PYG{p}{]}\PYG{p}{]}

\PYG{n}{We} \PYG{n}{can} \PYG{n}{get} \PYG{n}{a} \PYG{n}{handle} \PYG{k}{for} \PYG{n}{the} \PYG{n}{user} \PYG{n}{who} \PYG{n}{triggered} \PYG{n}{the} \PYG{n}{app}
\PYG{l+m+mi}{152152}\PYG{p}{:} \PYG{n}{Morten} \PYG{n}{Kallberg}
\end{Verbatim}

To start working, we will want to expand our permission scope for the trigger object so we can read and write data. The details of this process is the subject of the next section.
We end this section by demonstrating how one can easily obtain the so-called ``scope string,'' used when requesting further access, from the trigger object:

\begin{Verbatim}[commandchars=\\\{\}]
\PYG{n}{triggerObj} \PYG{o}{=} \PYG{n}{triggerType}\PYG{p}{[}\PYG{l+m+mi}{1}\PYG{p}{]}\PYG{p}{[}\PYG{o}{-}\PYG{l+m+mi}{1}\PYG{p}{]}
\PYG{k}{print} \PYG{l+s}{"}\PYG{l+s+se}{\PYGZbs{}n}\PYG{l+s}{The scope string for requesting write access to the trigger object is:}\PYG{l+s}{"}
\PYG{k}{print} \PYG{n}{triggerObj}\PYG{o}{.}\PYG{n}{getAccessStr}\PYG{p}{(}\PYG{n}{scope}\PYG{o}{=}\PYG{l+s}{'}\PYG{l+s}{write}\PYG{l+s}{'}\PYG{p}{)}
\end{Verbatim}

\begin{Verbatim}[commandchars=\\\{\}]
\PYG{n}{Output}\PYG{p}{[}\PYG{p}{]}\PYG{p}{:}

\PYG{n}{The} \PYG{n}{scope} \PYG{n}{string} \PYG{k}{for} \PYG{n}{requesting} \PYG{n}{write} \PYG{n}{access} \PYG{n}{to} \PYG{n}{the} \PYG{n}{trigger} \PYG{n+nb}{object} \PYG{o+ow}{is}\PYG{p}{:}
\PYG{n}{write} \PYG{n}{project} \PYG{l+m+mi}{89}
\end{Verbatim}


\section{Requesting an access-token for data browsing}
\label{Getting Started:requesting-an-access-token-for-data-browsing}
Here we demonstrate the basic BaseSpace authentication process. The work-flow outlined here is
\begin{enumerate}
\item {} 
Request of access to a specific data-scope

\item {} 
User approval of access request

\item {} 
Browsing data

\end{enumerate}

\begin{notice}{note}{Note:}
It will be useful if you are logged in to the BaseSpace web-site before launching this example to make the access grant procedure faster.
\end{notice}

Again we will start out by initializing a \code{BaseSpaceAuth} object:

\begin{Verbatim}[commandchars=\\\{\}]
\PYG{k+kn}{from} \PYG{n+nn}{BaseSpacePy.api.BaseSpaceAuth} \PYG{k+kn}{import} \PYG{n}{BaseSpaceAuth}
\PYG{k+kn}{import} \PYG{n+nn}{time}

\PYG{c}{\PYGZsh{} initialize an authentication object using the key and secret from your app}
\PYG{n}{client\PYGZus{}key}                 \PYG{o}{=} \PYG{o}{\textless{}}\PYG{n}{my} \PYG{n}{key}\PYG{o}{\textgreater{}}
\PYG{n}{client\PYGZus{}secret}              \PYG{o}{=} \PYG{o}{\textless{}}\PYG{n}{my} \PYG{n}{secret}\PYG{o}{\textgreater{}}
\PYG{n}{BaseSpaceUrl}               \PYG{o}{=} \PYG{l+s}{'}\PYG{l+s}{https://api.cloud-endor.illumina.com/}\PYG{l+s}{'}
\PYG{n}{version}                    \PYG{o}{=} \PYG{l+s}{'}\PYG{l+s}{v1pre2/}\PYG{l+s}{'}
\PYG{n}{BSauth} \PYG{o}{=} \PYG{n}{BaseSpaceAuth}\PYG{p}{(}\PYG{n}{client\PYGZus{}key}\PYG{p}{,}\PYG{n}{client\PYGZus{}secret}\PYG{p}{,}\PYG{n}{BaseSpaceUrl}\PYG{p}{,}\PYG{n}{version}\PYG{p}{)}
\end{Verbatim}

First get verification code and uri for scope `browse global'

\begin{Verbatim}[commandchars=\\\{\}]
\PYG{n}{deviceInfo} \PYG{o}{=} \PYG{n}{BSauth}\PYG{o}{.}\PYG{n}{getVerificationCode}\PYG{p}{(}\PYG{l+s}{'}\PYG{l+s}{browse global}\PYG{l+s}{'}\PYG{p}{)}
\end{Verbatim}

At this point the user must visit the verification uri to grant us access

\begin{Verbatim}[commandchars=\\\{\}]
\PYG{c}{\PYGZsh{}\PYGZsh{} PAUSE HERE}
\PYG{c}{\PYGZsh{} Have the user visit the verification uri to grant us access}
\PYG{k}{print} \PYG{l+s}{"}\PYG{l+s}{Please visit the uri within 30 seconds and grant access}\PYG{l+s}{"}
\PYG{k}{print} \PYG{n}{deviceInfo}\PYG{p}{[}\PYG{l+s}{'}\PYG{l+s}{verification\PYGZus{}with\PYGZus{}code\PYGZus{}uri}\PYG{l+s}{'}\PYG{p}{]}
\PYG{n}{time}\PYG{o}{.}\PYG{n}{sleep}\PYG{p}{(}\PYG{l+m+mi}{30}\PYG{p}{)}
\PYG{c}{\PYGZsh{}\PYGZsh{} PAUSE HERE}
\end{Verbatim}

\begin{Verbatim}[commandchars=\\\{\}]
Output[]:

Please visit the uri within 10 seconds and grant access
https://cloud-endor.illumina.com/oauth/device?code=\textless{}my device code\textgreater{}
\end{Verbatim}

There are two options for obtaining the access-token and instantiating a \code{BaseSpaceAPI} object:

\begin{Verbatim}[commandchars=\\\{\}]
\PYG{c}{\PYGZsh{} Get the access-token directly and instantiate an api yourself}
\PYG{c}{\PYGZsh{}token = BSauth.getAccessToken(deviceInfo['device\PYGZus{}code'])}
\PYG{c}{\PYGZsh{}print "My token " + str(token)}

\PYG{c}{\PYGZsh{} Alternatively we can generate an access-token and request a BaseSpaceApi instance}
\PYG{c}{\PYGZsh{} with the newly generated token in one step}
\PYG{n}{myAPI} \PYG{o}{=} \PYG{n}{BSauth}\PYG{o}{.}\PYG{n}{getBaseSpaceApi}\PYG{p}{(}\PYG{n}{deviceInfo}\PYG{p}{[}\PYG{l+s}{'}\PYG{l+s}{device\PYGZus{}code}\PYG{l+s}{'}\PYG{p}{]}\PYG{p}{)}
\PYG{k}{print} \PYG{n}{myAPI}
\end{Verbatim}

\begin{Verbatim}[commandchars=\\\{\}]
\PYG{n}{Output}\PYG{p}{[}\PYG{p}{]}\PYG{p}{:}

\PYG{n}{BaseSpaceAPI} \PYG{n}{instance} \PYG{o}{-} \PYG{n}{using} \PYG{n}{token}\PYG{o}{=}\PYG{o}{\textless{}}\PYG{n}{my} \PYG{n}{access} \PYG{n}{token}\PYG{o}{\textgreater{}}
\end{Verbatim}

At this point we can start using the \code{BaseSpaceAPI} instance to browse the available data for the current user, the details of this process is the subject of the next section. Here we will end with showing how the API object can be used to list all BaseSpace genome instances:

\begin{Verbatim}[commandchars=\\\{\}]
\PYG{c}{\PYGZsh{} We will get all available genomes with our new api!}
\PYG{n}{allGenomes}  \PYG{o}{=} \PYG{n}{myAPI}\PYG{o}{.}\PYG{n}{getAvailableGenomes}\PYG{p}{(}\PYG{p}{)}
\PYG{k}{print} \PYG{l+s}{"}\PYG{l+s+se}{\PYGZbs{}n}\PYG{l+s}{Genomes }\PYG{l+s+se}{\PYGZbs{}n}\PYG{l+s}{"} \PYG{o}{+} \PYG{n+nb}{str}\PYG{p}{(}\PYG{n}{allGenomes}\PYG{p}{)}
\end{Verbatim}

\begin{Verbatim}[commandchars=\\\{\}]
\PYG{n}{Output}\PYG{p}{[}\PYG{p}{]}\PYG{p}{:}

\PYG{n}{Genomes}
\PYG{p}{[}\PYG{n}{Arabidopsis} \PYG{n}{thaliana}\PYG{p}{,} \PYG{n}{Bos} \PYG{n}{Taurus}\PYG{p}{,} \PYG{n}{Escherichia} \PYG{n}{coli}\PYG{p}{,} \PYG{n}{Homo} \PYG{n}{sapiens}\PYG{p}{,} \PYG{n}{Mus} \PYG{n}{musculus}\PYG{p}{,} \PYG{n}{Phix}\PYG{p}{,}\PYGZbs{}
 \PYG{n}{Rhodobacter} \PYG{n}{sphaeroides}\PYG{p}{,} \PYG{n}{Rattus} \PYG{n}{norvegicus}\PYG{p}{,} \PYG{n}{Saccharomyces} \PYG{n}{cerevisiae}\PYG{p}{,} \PYG{n}{Staphylococcus} \PYG{n}{aureus}\PYG{p}{,} \PYG{n}{Bacillus} \PYG{n}{Cereus}\PYG{p}{]}
\end{Verbatim}


\section{Browsing data with \texttt{global browse}-access}
\label{Getting Started:browsing-data-with-global-browse-access}
This section demonstrates basic browsing of BaseSpace objects once an access-token for global browsing has been obtained. We will see how
objects can be retrieved using either the \code{BaseSpaceAPI} class or by use of method calls on related object instances (for example once
a \code{user} instance we can use it to retrieve all project belonging to that user).

First we will initialize a \code{BaseSpaceAPI} using our access-token for \code{global browse}:

\begin{Verbatim}[commandchars=\\\{\}]
\PYG{k+kn}{from} \PYG{n+nn}{BaseSpacePy.api.BaseSpaceAPI} \PYG{k+kn}{import} \PYG{n}{BaseSpaceAPI}

\PYG{c}{\PYGZsh{} REST server information and user access token}
\PYG{n}{server}          \PYG{o}{=} \PYG{l+s}{'}\PYG{l+s}{https://api.cloud-endor.illumina.com/}\PYG{l+s}{'}
\PYG{n}{version}         \PYG{o}{=} \PYG{l+s}{'}\PYG{l+s}{v1pre2}\PYG{l+s}{'}
\PYG{n}{access\PYGZus{}token}    \PYG{o}{=} \PYG{o}{\textless{}}\PYG{n}{my} \PYG{n}{access} \PYG{n}{token}\PYG{o}{\textgreater{}}

\PYG{c}{\PYGZsh{} First, create a client for making calls for this user session}
\PYG{n}{myAPI}   \PYG{o}{=} \PYG{n}{BaseSpaceAPI}\PYG{p}{(}\PYG{n}{AccessToken}\PYG{o}{=}\PYG{n}{access\PYGZus{}token}\PYG{p}{,}\PYG{n}{apiServer}\PYG{o}{=} \PYG{n}{server} \PYG{o}{+} \PYG{n}{version}\PYG{p}{)}
\end{Verbatim}

First we will try to retrieve a genome object:

\begin{Verbatim}[commandchars=\\\{\}]
\PYG{c}{\PYGZsh{} Now grab the genome with id=4}
\PYG{n}{myGenome}    \PYG{o}{=} \PYG{n}{myAPI}\PYG{o}{.}\PYG{n}{getGenomeById}\PYG{p}{(}\PYG{l+s}{'}\PYG{l+s}{4}\PYG{l+s}{'}\PYG{p}{)}
\PYG{k}{print} \PYG{l+s}{"}\PYG{l+s+se}{\PYGZbs{}n}\PYG{l+s}{The Genome is }\PYG{l+s}{"} \PYG{o}{+} \PYG{n+nb}{str}\PYG{p}{(}\PYG{n}{myGenome}\PYG{p}{)}
\PYG{k}{print} \PYG{l+s}{"}\PYG{l+s}{We can get more information from the genome object}\PYG{l+s}{"}
\PYG{k}{print} \PYG{l+s}{'}\PYG{l+s}{Id: }\PYG{l+s}{'} \PYG{o}{+} \PYG{n}{myGenome}\PYG{o}{.}\PYG{n}{Id}
\PYG{k}{print} \PYG{l+s}{'}\PYG{l+s}{Href: }\PYG{l+s}{'} \PYG{o}{+} \PYG{n}{myGenome}\PYG{o}{.}\PYG{n}{Href}
\PYG{k}{print} \PYG{l+s}{'}\PYG{l+s}{DisplayName: }\PYG{l+s}{'} \PYG{o}{+} \PYG{n}{myGenome}\PYG{o}{.}\PYG{n}{DisplayName}
\end{Verbatim}

\begin{Verbatim}[commandchars=\\\{\}]
Output[]:

The Genome is Homo sapiens
We can get more information from the genome object
Id: 4
Href: v1pre2/genomes/4
DisplayName: Homo Sapiens - UCSC (hg19)
\end{Verbatim}

Using a comparable method we can get a list of all available genomes:

\begin{Verbatim}[commandchars=\\\{\}]
\PYG{c}{\PYGZsh{} Get a list of all genomes}
\PYG{n}{allGenomes}  \PYG{o}{=} \PYG{n}{myAPI}\PYG{o}{.}\PYG{n}{getAvailableGenomes}\PYG{p}{(}\PYG{p}{)}
\PYG{k}{print} \PYG{l+s}{"}\PYG{l+s+se}{\PYGZbs{}n}\PYG{l+s}{Genomes }\PYG{l+s+se}{\PYGZbs{}n}\PYG{l+s}{"} \PYG{o}{+} \PYG{n+nb}{str}\PYG{p}{(}\PYG{n}{allGenomes}\PYG{p}{)}
\end{Verbatim}

\begin{Verbatim}[commandchars=\\\{\}]
\PYG{n}{Output}\PYG{p}{[}\PYG{p}{]}\PYG{p}{:}

\PYG{n}{Genomes}
\PYG{p}{[}\PYG{n}{Arabidopsis} \PYG{n}{thaliana}\PYG{p}{,} \PYG{n}{Bos} \PYG{n}{Taurus}\PYG{p}{,} \PYG{n}{Escherichia} \PYG{n}{coli}\PYG{p}{,} \PYG{n}{Homo} \PYG{n}{sapiens}\PYG{p}{,} \PYG{n}{Mus} \PYG{n}{musculus}\PYG{p}{,} \PYG{n}{Phix}\PYG{p}{,}\PYGZbs{}
 \PYG{n}{Rhodobacter} \PYG{n}{sphaeroides}\PYG{p}{,} \PYG{n}{Rattus} \PYG{n}{norvegicus}\PYG{p}{,} \PYG{n}{Saccharomyces} \PYG{n}{cerevisiae}\PYG{p}{,} \PYG{n}{Staphylococcus} \PYG{n}{aureus}\PYG{p}{,} \PYG{n}{Bacillus} \PYG{n}{Cereus}\PYG{p}{]}
\end{Verbatim}

Now, let us retrieve the \code{User} objects for the current user, and list all projects for this user:

\begin{Verbatim}[commandchars=\\\{\}]
\PYG{c}{\PYGZsh{} Take a look at the current user}
\PYG{n}{user}        \PYG{o}{=} \PYG{n}{myAPI}\PYG{o}{.}\PYG{n}{getUserById}\PYG{p}{(}\PYG{l+s}{'}\PYG{l+s}{current}\PYG{l+s}{'}\PYG{p}{)}
\PYG{k}{print} \PYG{l+s}{"}\PYG{l+s+se}{\PYGZbs{}n}\PYG{l+s}{The current user is }\PYG{l+s+se}{\PYGZbs{}n}\PYG{l+s}{"} \PYG{o}{+} \PYG{n+nb}{str}\PYG{p}{(}\PYG{n}{user}\PYG{p}{)}

\PYG{c}{\PYGZsh{} Now list the projects for this user}
\PYG{n}{myProjects}   \PYG{o}{=} \PYG{n}{myAPI}\PYG{o}{.}\PYG{n}{getProjectByUser}\PYG{p}{(}\PYG{l+s}{'}\PYG{l+s}{current}\PYG{l+s}{'}\PYG{p}{)}
\PYG{k}{print} \PYG{l+s}{"}\PYG{l+s+se}{\PYGZbs{}n}\PYG{l+s}{The projects for this user are }\PYG{l+s+se}{\PYGZbs{}n}\PYG{l+s}{"} \PYG{o}{+} \PYG{n+nb}{str}\PYG{p}{(}\PYG{n}{myProjects}\PYG{p}{)}
\end{Verbatim}

\begin{Verbatim}[commandchars=\\\{\}]
\PYG{n}{Output}\PYG{p}{[}\PYG{p}{]}\PYG{p}{:}

\PYG{n}{The} \PYG{n}{current} \PYG{n}{user} \PYG{o+ow}{is}
\PYG{l+m+mi}{152152}\PYG{p}{:} \PYG{n}{Morten} \PYG{n}{Kallberg}

\PYG{n}{The} \PYG{n}{projects} \PYG{k}{for} \PYG{n}{this} \PYG{n}{user} \PYG{n}{are}
\PYG{p}{[}\PYG{n}{HiSeq} \PYG{l+m+mi}{2500}\PYG{p}{,} \PYG{n}{Bolt}\PYG{p}{,} \PYG{n}{YourProject}\PYG{p}{,} \PYG{l+m+mi}{2}\PYG{n}{X151} \PYG{n}{Rhodobacter} \PYG{n}{Resequencing}\PYG{p}{,} \PYG{n}{EColi} \PYG{n}{resequencing}\PYG{p}{]}
\end{Verbatim}

We can also achieve this by making a call using the \code{user} instance. Notice that these calls take an instance of \code{BaseSpaceAPI} with apporpriate
priviliges to complete the transaction as parameter, this true for all retrieval method calls made on data objects:

\begin{Verbatim}[commandchars=\\\{\}]
\PYG{n}{myProjects2} \PYG{o}{=} \PYG{n}{user}\PYG{o}{.}\PYG{n}{getProjects}\PYG{p}{(}\PYG{n}{myAPI}\PYG{p}{)}
\PYG{k}{print} \PYG{l+s}{"}\PYG{l+s+se}{\PYGZbs{}n}\PYG{l+s}{Projects retrieved from the user instance }\PYG{l+s+se}{\PYGZbs{}n}\PYG{l+s}{"} \PYG{o}{+} \PYG{n+nb}{str}\PYG{p}{(}\PYG{n}{myProjects2}\PYG{p}{)}

\PYG{c}{\PYGZsh{} List the runs available for the current user}
\PYG{n}{runs} \PYG{o}{=} \PYG{n}{user}\PYG{o}{.}\PYG{n}{getRuns}\PYG{p}{(}\PYG{n}{myAPI}\PYG{p}{)}
\PYG{k}{print} \PYG{l+s}{"}\PYG{l+s+se}{\PYGZbs{}n}\PYG{l+s}{The runs for this user are }\PYG{l+s+se}{\PYGZbs{}n}\PYG{l+s}{"} \PYG{o}{+} \PYG{n+nb}{str}\PYG{p}{(}\PYG{n}{runs}\PYG{p}{)}
\end{Verbatim}

\begin{Verbatim}[commandchars=\\\{\}]
Output[]:

Projects retrieved from the user instance
[HiSeq 2500, Bolt, YourProject, 2X151 Rhodobacter Resequencing, EColi resequencing]

The runs for this user are
[2X151 Rhodobacter Resequencing, 2x26 Validation Human 4-Plex, EColi resequencing]
\end{Verbatim}

In the same manner we can get a list of accessible user runs:

\begin{Verbatim}[commandchars=\\\{\}]
\PYG{n}{runs} \PYG{o}{=} \PYG{n}{user}\PYG{o}{.}\PYG{n}{getRuns}\PYG{p}{(}\PYG{n}{myAPI}\PYG{p}{)}
\PYG{k}{print} \PYG{l+s}{"}\PYG{l+s+se}{\PYGZbs{}n}\PYG{l+s}{Runs retrieved from user instance }\PYG{l+s+se}{\PYGZbs{}n}\PYG{l+s}{"} \PYG{o}{+} \PYG{n+nb}{str}\PYG{p}{(}\PYG{n}{runs}\PYG{p}{)}
\end{Verbatim}

\begin{Verbatim}[commandchars=\\\{\}]
Output[]:

Runs retrieved from user instance
[2X151 Rhodobacter Resequencing, 2x26 Validation Human 4-Plex, EColi resequencing]
\end{Verbatim}


\section{Accessing file-trees and querying BAM/VCF files}
\label{Getting Started:accessing-file-trees-and-querying-bam-vcf-files}
In this section we demonstrate how to access samples and analysis from a projects and how to work with the available file data for such instances.
In addition, we take a look at some of the special queuring methods associated with BAM- and VCF-files.

Again, start out by initializing a \code{BaseSpacePy} instance and retrieving all projects belonging to the current user:

\begin{Verbatim}[commandchars=\\\{\}]
\PYG{c}{\PYGZsh{} First, create a client for making calls for this user session}
\PYG{n}{myAPI}   \PYG{o}{=} \PYG{n}{BaseSpaceAPI}\PYG{p}{(}\PYG{n}{AccessToken}\PYG{o}{=}\PYG{n}{access\PYGZus{}token}\PYG{p}{,}\PYG{n}{apiServer}\PYG{o}{=} \PYG{n}{server} \PYG{o}{+} \PYG{n}{version}\PYG{p}{)}
\PYG{n}{user}        \PYG{o}{=} \PYG{n}{myAPI}\PYG{o}{.}\PYG{n}{getUserById}\PYG{p}{(}\PYG{l+s}{'}\PYG{l+s}{current}\PYG{l+s}{'}\PYG{p}{)}
\PYG{n}{myProjects}   \PYG{o}{=} \PYG{n}{myAPI}\PYG{o}{.}\PYG{n}{getProjectByUser}\PYG{p}{(}\PYG{l+s}{'}\PYG{l+s}{current}\PYG{l+s}{'}\PYG{p}{)}
\end{Verbatim}

Now we can list all the analyses and samples for these projects

\begin{Verbatim}[commandchars=\\\{\}]
\PYG{k}{for} \PYG{n}{singleProject} \PYG{o+ow}{in} \PYG{n}{myProjects}\PYG{p}{:}
    \PYG{k}{print} \PYG{l+s}{"}\PYG{l+s}{\PYGZsh{} }\PYG{l+s}{"} \PYG{o}{+} \PYG{n+nb}{str}\PYG{p}{(}\PYG{n}{singleProject}\PYG{p}{)}
    \PYG{n}{analyses} \PYG{o}{=} \PYG{n}{singleProject}\PYG{o}{.}\PYG{n}{getAnalyses}\PYG{p}{(}\PYG{n}{myAPI}\PYG{p}{)}
    \PYG{k}{print} \PYG{l+s}{"}\PYG{l+s}{    The analysis for project }\PYG{l+s}{"} \PYG{o}{+} \PYG{n+nb}{str}\PYG{p}{(}\PYG{n}{singleProject}\PYG{p}{)} \PYG{o}{+} \PYG{l+s}{"}\PYG{l+s}{ are }\PYG{l+s+se}{\PYGZbs{}n}\PYG{l+s+se}{\PYGZbs{}t}\PYG{l+s}{"} \PYG{o}{+} \PYG{n+nb}{str}\PYG{p}{(}\PYG{n}{analyses}\PYG{p}{)}
    \PYG{n}{samples} \PYG{o}{=} \PYG{n}{singleProject}\PYG{o}{.}\PYG{n}{getSamples}\PYG{p}{(}\PYG{n}{myAPI}\PYG{p}{)}
    \PYG{k}{print} \PYG{l+s}{"}\PYG{l+s}{    The samples for project }\PYG{l+s}{"} \PYG{o}{+} \PYG{n+nb}{str}\PYG{p}{(}\PYG{n}{singleProject}\PYG{p}{)} \PYG{o}{+} \PYG{l+s}{"}\PYG{l+s}{ are }\PYG{l+s+se}{\PYGZbs{}n}\PYG{l+s+se}{\PYGZbs{}t}\PYG{l+s}{"} \PYG{o}{+} \PYG{n+nb}{str}\PYG{p}{(}\PYG{n}{samples}\PYG{p}{)}
\end{Verbatim}

\begin{Verbatim}[commandchars=\\\{\}]
\PYG{n}{Output}\PYG{p}{[}\PYG{p}{]}\PYG{p}{:}

\PYG{c}{\PYGZsh{} HiSeq 2500}
    \PYG{n}{The} \PYG{n}{analysis} \PYG{k}{for} \PYG{n}{project} \PYG{n}{HiSeq} \PYG{l+m+mi}{2500} \PYG{n}{are}
        \PYG{p}{[}\PYG{n}{Resequencing} \PYG{o}{-} \PYG{n}{Completed}\PYG{p}{]}
    \PYG{n}{The} \PYG{n}{samples} \PYG{k}{for} \PYG{n}{project} \PYG{n}{HiSeq} \PYG{l+m+mi}{2500} \PYG{n}{are}
        \PYG{p}{[}\PYG{n}{NA18507}\PYG{p}{]}
\PYG{c}{\PYGZsh{} Bolt}
    \PYG{n}{The} \PYG{n}{analysis} \PYG{k}{for} \PYG{n}{project} \PYG{n}{Bolt} \PYG{n}{are}
        \PYG{p}{[}\PYG{n}{Amplicon} \PYG{o}{-} \PYG{n}{Completed}\PYG{p}{,} \PYG{n}{Amplicon} \PYG{o}{-} \PYG{n}{Completed}\PYG{p}{,} \PYG{n}{Amplicon} \PYG{o}{.}\PYG{o}{.}\PYG{o}{.}
    \PYG{n}{The} \PYG{n}{samples} \PYG{k}{for} \PYG{n}{project} \PYG{n}{Bolt} \PYG{n}{are}
        \PYG{p}{[}\PYG{n}{sample\PYGZus{}1}\PYG{p}{,} \PYG{n}{sample\PYGZus{}2}\PYG{p}{,} \PYG{n}{sample\PYGZus{}3}\PYG{p}{,} \PYG{o}{.}\PYG{o}{.}\PYG{o}{.}

\PYG{o}{.}\PYG{o}{.}\PYG{o}{.}\PYG{o}{.}\PYG{o}{.}\PYG{o}{.}
\end{Verbatim}

We'll take a further look at the files belonging to the sample from the last project in the loop above:

\begin{Verbatim}[commandchars=\\\{\}]
\PYG{k}{for} \PYG{n}{s} \PYG{o+ow}{in} \PYG{n}{samples}\PYG{p}{:}
    \PYG{k}{print} \PYG{l+s}{"}\PYG{l+s}{Sample }\PYG{l+s}{"} \PYG{o}{+} \PYG{n+nb}{str}\PYG{p}{(}\PYG{n}{s}\PYG{p}{)}
    \PYG{n}{ff} \PYG{o}{=} \PYG{n}{s}\PYG{o}{.}\PYG{n}{getFiles}\PYG{p}{(}\PYG{n}{myAPI}\PYG{p}{)}
    \PYG{k}{print} \PYG{n}{ff}
\end{Verbatim}

\begin{Verbatim}[commandchars=\\\{\}]
\PYG{n}{Output}\PYG{p}{[}\PYG{p}{]}\PYG{p}{:}

\PYG{n}{Sample} \PYG{n}{Ecoli}
\PYG{p}{[}\PYG{n}{s\PYGZus{}G1\PYGZus{}L001\PYGZus{}R1\PYGZus{}001}\PYG{o}{.}\PYG{n}{fastq}\PYG{o}{.}\PYG{l+m+mf}{1.}\PYG{n}{gz}\PYG{p}{,} \PYG{n}{s\PYGZus{}G1\PYGZus{}L001\PYGZus{}R1\PYGZus{}002}\PYG{o}{.}\PYG{n}{fastq}\PYG{o}{.}\PYG{l+m+mf}{1.}\PYG{n}{gz}\PYG{p}{,} \PYG{n}{s\PYGZus{}G1\PYGZus{}L001\PYGZus{}R2\PYGZus{}001}\PYG{o}{.}\PYG{n}{fastq}\PYG{o}{.}\PYG{l+m+mf}{1.}\PYG{n}{gz}\PYG{p}{,} \PYG{n}{s\PYGZus{}G1\PYGZus{}L001\PYGZus{}R2\PYGZus{}002}\PYG{o}{.}\PYG{n}{fastq}\PYG{o}{.}\PYG{l+m+mf}{1.}\PYG{n}{gz}\PYG{p}{]}
\end{Verbatim}

Now, have a look at some of the methods calls specific to \code{Bam} and \code{VCF} files. First, we will get a \code{Bam}-file and then retrieve the coverage information available for chromosome 2 between positions 1 and 20000:

\begin{Verbatim}[commandchars=\\\{\}]
\PYG{c}{\PYGZsh{} Now do some work with files}
\PYG{c}{\PYGZsh{} we'll grab a BAM by id and get the coverage for an interval + accompanying meta-data}
\PYG{n}{myBam} \PYG{o}{=} \PYG{n}{myAPI}\PYG{o}{.}\PYG{n}{getFileById}\PYG{p}{(}\PYG{l+s}{'}\PYG{l+s}{2150156}\PYG{l+s}{'}\PYG{p}{)}
\PYG{k}{print} \PYG{n}{myBam}
\PYG{n}{cov}     \PYG{o}{=} \PYG{n}{myBam}\PYG{o}{.}\PYG{n}{getIntervalCoverage}\PYG{p}{(}\PYG{n}{myAPI}\PYG{p}{,}\PYG{l+s}{'}\PYG{l+s}{chr2}\PYG{l+s}{'}\PYG{p}{,}\PYG{l+s}{'}\PYG{l+s}{1}\PYG{l+s}{'}\PYG{p}{,}\PYG{l+s}{'}\PYG{l+s}{20000}\PYG{l+s}{'}\PYG{p}{)}
\PYG{k}{print} \PYG{n}{cov}
\PYG{n}{covMeta} \PYG{o}{=} \PYG{n}{myBam}\PYG{o}{.}\PYG{n}{getCoverageMeta}\PYG{p}{(}\PYG{n}{myAPI}\PYG{p}{,}\PYG{l+s}{'}\PYG{l+s}{chr2}\PYG{l+s}{'}\PYG{p}{)}
\PYG{k}{print} \PYG{n}{covMeta}
\end{Verbatim}

\begin{Verbatim}[commandchars=\\\{\}]
\PYG{n}{Output}\PYG{p}{[}\PYG{p}{]}\PYG{p}{:}

\PYG{n+nb}{sorted}\PYG{o}{.}\PYG{n}{bam}
\PYG{n}{Chrchr2}\PYG{p}{:} \PYG{l+m+mi}{1}\PYG{o}{-}\PYG{l+m+mi}{20096}\PYG{p}{:} \PYG{n}{BucketSize}\PYG{o}{=}\PYG{l+m+mi}{16}
\PYG{n}{CoverageMeta}\PYG{p}{:} \PYG{n+nb}{max}\PYG{o}{=}\PYG{l+m+mi}{20483} \PYG{n}{gran}\PYG{o}{=}\PYG{l+m+mi}{128}
\end{Verbatim}

For \code{VCF}-files we can filter variant calls based on chromosome and location as well:

\begin{Verbatim}[commandchars=\\\{\}]
\PYG{c}{\PYGZsh{} and a vcf file}
\PYG{n}{myVCF} \PYG{o}{=} \PYG{n}{myAPI}\PYG{o}{.}\PYG{n}{getFileById}\PYG{p}{(}\PYG{l+s}{'}\PYG{l+s}{2150158}\PYG{l+s}{'}\PYG{p}{)}
\PYG{c}{\PYGZsh{} Get the variant meta info}
\PYG{n}{varMeta} \PYG{o}{=} \PYG{n}{myVCF}\PYG{o}{.}\PYG{n}{getVariantMeta}\PYG{p}{(}\PYG{n}{myAPI}\PYG{p}{)}
\PYG{k}{print} \PYG{n}{varMeta}
\PYG{n}{var}     \PYG{o}{=} \PYG{n}{myVCF}\PYG{o}{.}\PYG{n}{filterVariant}\PYG{p}{(}\PYG{n}{myAPI}\PYG{p}{,}\PYG{l+s}{'}\PYG{l+s}{2}\PYG{l+s}{'}\PYG{p}{,}\PYG{l+s}{'}\PYG{l+s}{1}\PYG{l+s}{'}\PYG{p}{,} \PYG{l+s}{'}\PYG{l+s}{11000}\PYG{l+s}{'}\PYG{p}{)}
\PYG{k}{print} \PYG{n}{var}
\end{Verbatim}

\begin{Verbatim}[commandchars=\\\{\}]
\PYG{n}{Output}\PYG{p}{[}\PYG{p}{]}\PYG{p}{:}

\PYG{n}{VariantHeader}\PYG{p}{:} \PYG{n}{SampleCount}\PYG{o}{=}\PYG{l+m+mi}{1}
\PYG{p}{[}\PYG{n}{Variant} \PYG{o}{-} \PYG{n}{chr2}\PYG{p}{:} \PYG{l+m+mi}{10236} \PYG{n+nb}{id}\PYG{o}{=}\PYG{p}{[}\PYG{l+s}{'}\PYG{l+s}{.}\PYG{l+s}{'}\PYG{p}{]}\PYG{p}{,} \PYG{n}{Variant} \PYG{o}{-} \PYG{n}{chr2}\PYG{p}{:} \PYG{l+m+mi}{10249} \PYG{n+nb}{id}\PYG{o}{=}\PYG{p}{[}\PYG{l+s}{'}\PYG{l+s}{.}\PYG{l+s}{'}\PYG{p}{]}\PYG{p}{,} \PYG{o}{.}\PYG{o}{.}\PYG{o}{.}\PYG{o}{.}
\end{Verbatim}


\section{Creating an analysis and uploading results}
\label{Getting Started:creating-an-analysis-and-uploading-results}
In this section we will see how to create a new analysis object, change its state
and upload result files to it as well as retrieve files from it.

First, create a client for making calls for this user session:

\begin{Verbatim}[commandchars=\\\{\}]
\PYG{n}{myBaseSpaceAPI}   \PYG{o}{=} \PYG{n}{BaseSpaceAPI}\PYG{p}{(}\PYG{n}{AccessToken}\PYG{o}{=}\PYG{n}{access\PYGZus{}token}\PYG{p}{,}\PYG{n}{apiServer}\PYG{o}{=} \PYG{n}{server} \PYG{o}{+} \PYG{n}{version}\PYG{p}{)}
\PYG{c}{\PYGZsh{}}
\PYG{c}{\PYGZsh{}\PYGZsh{} Now we'll do some work of our own. First get a project to work on}
\PYG{c}{\PYGZsh{}\PYGZsh{} we'll need write permission, for the project we are working on}
\PYG{c}{\PYGZsh{}\PYGZsh{} meaning we will need get a new token and instantiate a new BaseSpaceAPI}
\PYG{n}{p} \PYG{o}{=} \PYG{n}{myBaseSpaceAPI}\PYG{o}{.}\PYG{n}{getProjectById}\PYG{p}{(}\PYG{l+s}{'}\PYG{l+s}{89}\PYG{l+s}{'}\PYG{p}{)}
\PYG{c}{\PYGZsh{} A short-cut for getting a scope string if we already have a project-instance:}
\PYG{k}{print} \PYG{n}{p}\PYG{o}{.}\PYG{n}{getAccessStr}\PYG{p}{(}\PYG{n}{scope}\PYG{o}{=}\PYG{l+s}{'}\PYG{l+s}{write}\PYG{l+s}{'}\PYG{p}{)}
\PYG{c}{\PYGZsh{} or simply}
\PYG{n}{p}\PYG{o}{.}\PYG{n}{getAccessStr}\PYG{p}{(}\PYG{p}{)}
\end{Verbatim}

\begin{Verbatim}[commandchars=\\\{\}]
\PYG{n}{Output}\PYG{p}{[}\PYG{p}{]}\PYG{p}{:}

\PYG{n}{write} \PYG{n}{project} \PYG{l+m+mi}{89}
\end{Verbatim}

Assuming we now have write access for the project, we will list the current analyses for the project:

\begin{Verbatim}[commandchars=\\\{\}]
\PYG{n}{ana} \PYG{o}{=} \PYG{n}{p}\PYG{o}{.}\PYG{n}{getAnalyses}\PYG{p}{(}\PYG{n}{myBaseSpaceAPI}\PYG{p}{)}
\PYG{k}{print} \PYG{l+s}{"}\PYG{l+s+se}{\PYGZbs{}n}\PYG{l+s}{The current analyses are }\PYG{l+s+se}{\PYGZbs{}n}\PYG{l+s}{"} \PYG{o}{+} \PYG{n+nb}{str}\PYG{p}{(}\PYG{n}{ana}\PYG{p}{)}
\end{Verbatim}

\begin{Verbatim}[commandchars=\\\{\}]
\PYG{n}{Output}\PYG{p}{[}\PYG{p}{]}\PYG{p}{:}

\PYG{n}{The} \PYG{n}{current} \PYG{n}{analyses} \PYG{n}{are}
\PYG{p}{[}\PYG{n}{Results} \PYG{k}{for} \PYG{n}{sample} \PYG{l+m+mi}{123} \PYG{o}{-} \PYG{n}{Working}\PYG{p}{,} \PYG{n}{Results} \PYG{k}{for} \PYG{n}{sample} \PYG{l+m+mi}{124} \PYG{o}{-} \PYG{n}{Working}\PYG{o}{.}\PYG{o}{.}\PYG{o}{.}
\end{Verbatim}

To create an analysis for a project, simply give the name and description to the \code{createAnalysis}-method:

\begin{Verbatim}[commandchars=\\\{\}]
\PYG{n}{analysis} \PYG{o}{=} \PYG{n}{p}\PYG{o}{.}\PYG{n}{createAnalysis}\PYG{p}{(}\PYG{n}{myBaseSpaceAPI}\PYG{p}{,}\PYG{l+s}{"}\PYG{l+s}{My very first analysis!!}\PYG{l+s}{"}\PYG{p}{,}\PYG{l+s}{"}\PYG{l+s}{This is my analysis}\PYG{l+s}{"}\PYG{p}{)}
\PYG{k}{print} \PYG{l+s}{"}\PYG{l+s+se}{\PYGZbs{}n}\PYG{l+s}{Some info about our new analysis}\PYG{l+s}{"}
\PYG{k}{print} \PYG{n}{analysis}
\PYG{k}{print} \PYG{n}{analysis}\PYG{o}{.}\PYG{n}{Id}
\PYG{k}{print} \PYG{n}{analysis}\PYG{o}{.}\PYG{n}{Status}
\PYG{c}{\PYGZsh{} we can change the status of out analysis and add a status-summary as follows}
\PYG{n}{analysis}\PYG{o}{.}\PYG{n}{setStatus}\PYG{p}{(}\PYG{n}{myBaseSpaceAPI}\PYG{p}{,}\PYG{l+s}{'}\PYG{l+s}{completed}\PYG{l+s}{'}\PYG{p}{,}\PYG{l+s}{"}\PYG{l+s}{We worked hard.}\PYG{l+s}{"}\PYG{p}{)}
\PYG{k}{print} \PYG{l+s}{"}\PYG{l+s+se}{\PYGZbs{}n}\PYG{l+s}{After a change of status we get}\PYG{l+s+se}{\PYGZbs{}n}\PYG{l+s}{"} \PYG{o}{+} \PYG{n+nb}{str}\PYG{p}{(}\PYG{n}{analysis}\PYG{p}{)}

\PYG{c}{\PYGZsh{}\PYGZsh{}\PYGZsh{} List the analyses again and see if our new object shows up}
\PYG{n}{ana} \PYG{o}{=} \PYG{n}{p}\PYG{o}{.}\PYG{n}{getAnalyses}\PYG{p}{(}\PYG{n}{myBaseSpaceAPI}\PYG{p}{)}
\PYG{k}{print} \PYG{l+s}{"}\PYG{l+s+se}{\PYGZbs{}n}\PYG{l+s}{The updated analyses are }\PYG{l+s+se}{\PYGZbs{}n}\PYG{l+s}{"} \PYG{o}{+} \PYG{n+nb}{str}\PYG{p}{(}\PYG{n}{ana}\PYG{p}{)}
\end{Verbatim}

\begin{Verbatim}[commandchars=\\\{\}]
Output[]:

Some info about our new analysis
My very first analysis!! - Working
94094
Working

After a change of status we get
My very first analysis!! - Completed

The updated analyses are
[Results for sample 123 - Working, Results for sample 124 - Working, Results for sample 124 - Working, ...
\end{Verbatim}

Now we will make another analysis and try to upload some files to it:

\begin{Verbatim}[commandchars=\\\{\}]
\PYG{n}{analysis2} \PYG{o}{=} \PYG{n}{p}\PYG{o}{.}\PYG{n}{createAnalysis}\PYG{p}{(}\PYG{n}{myBaseSpaceAPI}\PYG{p}{,}\PYG{l+s}{"}\PYG{l+s}{My second analysis}\PYG{l+s}{"}\PYG{p}{,}\PYG{l+s}{"}\PYG{l+s}{This one I will upload to}\PYG{l+s}{"}\PYG{p}{)}
\PYG{n}{analysis2}\PYG{o}{.}\PYG{n}{uploadFile}\PYG{p}{(}\PYG{n}{myBaseSpaceAPI}\PYG{p}{,} \PYG{l+s}{'}\PYG{l+s}{/my/file/dir/testFile2.txt}\PYG{l+s}{'}\PYG{p}{,} \PYG{l+s}{'}\PYG{l+s}{BaseSpaceTestFile.txt}\PYG{l+s}{'}\PYG{p}{,} \PYG{l+s}{'}\PYG{l+s}{/mydir/}\PYG{l+s}{'}\PYG{p}{,} \PYG{l+s}{'}\PYG{l+s}{text/plain}\PYG{l+s}{'}\PYG{p}{)}
\PYG{k}{print} \PYG{l+s}{"}\PYG{l+s+se}{\PYGZbs{}n}\PYG{l+s}{My analysis number 2 }\PYG{l+s+se}{\PYGZbs{}n}\PYG{l+s}{"} \PYG{o}{+} \PYG{n+nb}{str}\PYG{p}{(}\PYG{n}{analysis2}\PYG{p}{)}
\PYG{c}{\PYGZsh{}}
\PYG{c}{\PYGZsh{}\PYGZsh{} Check to see if our new file made it}
\PYG{n}{analysisFiles} \PYG{o}{=} \PYG{n}{analysis2}\PYG{o}{.}\PYG{n}{getFiles}\PYG{p}{(}\PYG{n}{myBaseSpaceAPI}\PYG{p}{)}
\PYG{k}{print} \PYG{l+s}{"}\PYG{l+s+se}{\PYGZbs{}n}\PYG{l+s}{These are the files in the analysis}\PYG{l+s}{"}
\PYG{k}{print} \PYG{n}{analysisFiles}
\PYG{n}{f} \PYG{o}{=} \PYG{n}{analysisFiles}\PYG{p}{[}\PYG{o}{-}\PYG{l+m+mi}{1}\PYG{p}{]}
\end{Verbatim}

\begin{Verbatim}[commandchars=\\\{\}]
\PYG{n}{Output}\PYG{p}{[}\PYG{p}{]}\PYG{p}{:}

\PYG{n}{My} \PYG{n}{analysis} \PYG{n}{number} \PYG{l+m+mi}{2}
\PYG{n}{My} \PYG{n}{second} \PYG{n}{analysis} \PYG{o}{-} \PYG{n}{Working}

\PYG{n}{These} \PYG{n}{are} \PYG{n}{the} \PYG{n}{files} \PYG{o+ow}{in} \PYG{n}{the} \PYG{n}{analysis}
\PYG{p}{[}\PYG{n}{BaseSpaceTestFile}\PYG{o}{.}\PYG{n}{txt}\PYG{p}{]}
\end{Verbatim}

We can even download our newly uploaded file in the following manner:

\begin{Verbatim}[commandchars=\\\{\}]
\PYG{n}{f}\PYG{o}{.}\PYG{n}{downloadFile}\PYG{p}{(}\PYG{n}{myBaseSpaceAPI}\PYG{p}{,}\PYG{l+s}{'}\PYG{l+s}{/path/to/place/file/in/}\PYG{l+s}{'}\PYG{p}{)}
\end{Verbatim}


\chapter{Advanced examples}
\label{Advanced examples:advanced-examples}\label{Advanced examples::doc}

\section{Multi-part upload}
\label{Advanced examples:multi-part-upload}
To-do


\section{Example server}
\label{Advanced examples:example-server}
To-do


\chapter{Available modules}
\label{Available modules:available-modules}\label{Available modules::doc}

\section{API}
\label{Available modules:api}\index{BaseSpaceAPI (class in BaseSpacePy.api.BaseSpaceAPI)}

\begin{fulllineitems}
\phantomsection\label{Available modules:BaseSpacePy.api.BaseSpaceAPI.BaseSpaceAPI}\pysiglinewithargsret{\strong{class }\code{BaseSpacePy.api.BaseSpaceAPI.}\bfcode{BaseSpaceAPI}}{\emph{AccessToken}, \emph{apiServer}}{}
The main API class used for all communication with with the REST server
\index{analysisFileUpload() (BaseSpacePy.api.BaseSpaceAPI.BaseSpaceAPI method)}

\begin{fulllineitems}
\phantomsection\label{Available modules:BaseSpacePy.api.BaseSpaceAPI.BaseSpaceAPI.analysisFileUpload}\pysiglinewithargsret{\bfcode{analysisFileUpload}}{\emph{Id}, \emph{localPath}, \emph{fileName}, \emph{directory}, \emph{contentType}}{}
Uploads a file associated with an analysis to BaseSpace and returns the corresponding file object
\begin{quote}\begin{description}
\item[{Parameters}] \leavevmode\begin{itemize}
\item {} 
\textbf{Id} -- Analysis id.

\item {} 
\textbf{localPath} -- The local path to the file to be uploaded.

\item {} 
\textbf{fileName} -- The desired filename in the Analysis folder on the BaseSpace server.

\item {} 
\textbf{directory} -- The directory the file should be placed in.

\item {} 
\textbf{contentType} -- The content-type of the file.

\end{itemize}

\end{description}\end{quote}

\end{fulllineitems}

\index{createAnalyses() (BaseSpacePy.api.BaseSpaceAPI.BaseSpaceAPI method)}

\begin{fulllineitems}
\phantomsection\label{Available modules:BaseSpacePy.api.BaseSpaceAPI.BaseSpaceAPI.createAnalyses}\pysiglinewithargsret{\bfcode{createAnalyses}}{\emph{Id}, \emph{name}, \emph{desc}}{}
Create an analysis object
\begin{quote}\begin{description}
\item[{Parameters}] \leavevmode\begin{itemize}
\item {} 
\textbf{Id} -- The id for the project in which the analysis is to be added

\item {} 
\textbf{name} -- The name of the analysis

\item {} 
\textbf{desc} -- A describtion of the analysis

\end{itemize}

\end{description}\end{quote}

\end{fulllineitems}

\index{fileDownload() (BaseSpacePy.api.BaseSpaceAPI.BaseSpaceAPI method)}

\begin{fulllineitems}
\phantomsection\label{Available modules:BaseSpacePy.api.BaseSpaceAPI.BaseSpaceAPI.fileDownload}\pysiglinewithargsret{\bfcode{fileDownload}}{\emph{Id}, \emph{localDir}, \emph{name}, \emph{range=}\optional{}}{}
Downloads a BaseSpace file to a local directory
\begin{quote}\begin{description}
\item[{Parameters}] \leavevmode\begin{itemize}
\item {} 
\textbf{Id} -- The file id

\item {} 
\textbf{localDir} -- The local directory to place the file in

\item {} 
\textbf{name} -- The name of the local file

\item {} 
\textbf{range} -- (Optional) The byte range of the file to retrieve (not yet implemented)

\end{itemize}

\end{description}\end{quote}

\end{fulllineitems}

\index{filterVariantSet() (BaseSpacePy.api.BaseSpaceAPI.BaseSpaceAPI method)}

\begin{fulllineitems}
\phantomsection\label{Available modules:BaseSpacePy.api.BaseSpaceAPI.BaseSpaceAPI.filterVariantSet}\pysiglinewithargsret{\bfcode{filterVariantSet}}{\emph{Id}, \emph{Chrom}, \emph{StartPos}, \emph{EndPos}, \emph{Format}, \emph{queryPars=\{`Limit': `100'}, \emph{`SortBy': `Position'}, \emph{`SortDir': `Asc'}, \emph{`Offset': `0'\}}}{}
List the variants in a set of variants. Maximum returned records is 1000
\begin{quote}\begin{description}
\item[{Parameters}] \leavevmode\begin{itemize}
\item {} 
\textbf{Id} -- The id of the variant file

\item {} 
\textbf{Chrom} -- The chromosome of interest

\item {} 
\textbf{StartPos} -- The start position of the sequence of interest

\item {} 
\textbf{EndPos} -- The start position of the sequence of interest

\item {} 
\textbf{Format} -- Set to `vcf' to get the results as lines in VCF format

\item {} 
\textbf{queryPars} -- An (optional) object of type QueryParameters for custom sorting and filtering

\end{itemize}

\end{description}\end{quote}

\end{fulllineitems}

\index{getAccessToken() (BaseSpacePy.api.BaseSpaceAPI.BaseSpaceAPI method)}

\begin{fulllineitems}
\phantomsection\label{Available modules:BaseSpacePy.api.BaseSpaceAPI.BaseSpaceAPI.getAccessToken}\pysiglinewithargsret{\bfcode{getAccessToken}}{}{}
Returns the access-token that was used to initialize the BaseSpaceAPI object.

\end{fulllineitems}

\index{getAccessibleRunsByUser() (BaseSpacePy.api.BaseSpaceAPI.BaseSpaceAPI method)}

\begin{fulllineitems}
\phantomsection\label{Available modules:BaseSpacePy.api.BaseSpaceAPI.BaseSpaceAPI.getAccessibleRunsByUser}\pysiglinewithargsret{\bfcode{getAccessibleRunsByUser}}{\emph{Id}, \emph{queryPars=\{`Limit': `100'}, \emph{`SortBy': `Id'}, \emph{`SortDir': `Asc'}, \emph{`Offset': `0'\}}}{}
Returns a list of accessible runs for the User with id=Id
\begin{quote}\begin{description}
\item[{Parameters}] \leavevmode\begin{itemize}
\item {} 
\textbf{Id} -- An user id

\item {} 
\textbf{queryPars} -- An (optional) object of type QueryParameters for custom sorting and filtering

\end{itemize}

\end{description}\end{quote}

\end{fulllineitems}

\index{getAnalysisById() (BaseSpacePy.api.BaseSpaceAPI.BaseSpaceAPI method)}

\begin{fulllineitems}
\phantomsection\label{Available modules:BaseSpacePy.api.BaseSpaceAPI.BaseSpaceAPI.getAnalysisById}\pysiglinewithargsret{\bfcode{getAnalysisById}}{\emph{Id}}{}
Returns an Analysis object corresponding to Id
\begin{quote}\begin{description}
\item[{Parameters}] \leavevmode
\textbf{Id} -- The Id of the Analysis

\end{description}\end{quote}

\end{fulllineitems}

\index{getAnalysisByProject() (BaseSpacePy.api.BaseSpaceAPI.BaseSpaceAPI method)}

\begin{fulllineitems}
\phantomsection\label{Available modules:BaseSpacePy.api.BaseSpaceAPI.BaseSpaceAPI.getAnalysisByProject}\pysiglinewithargsret{\bfcode{getAnalysisByProject}}{\emph{Id}, \emph{queryPars=\{`Limit': `100'}, \emph{`SortBy': `Id'}, \emph{`SortDir': `Asc'}, \emph{`Offset': `0'\}}}{}
Returns a list of Analysis object associated with the project with Id
\begin{quote}\begin{description}
\item[{Parameters}] \leavevmode\begin{itemize}
\item {} 
\textbf{Id} -- The project id

\item {} 
\textbf{queryPars} -- An (optional) object of type QueryParameters for custom sorting and filtering

\end{itemize}

\end{description}\end{quote}

\end{fulllineitems}

\index{getAnalysisFiles() (BaseSpacePy.api.BaseSpaceAPI.BaseSpaceAPI method)}

\begin{fulllineitems}
\phantomsection\label{Available modules:BaseSpacePy.api.BaseSpaceAPI.BaseSpaceAPI.getAnalysisFiles}\pysiglinewithargsret{\bfcode{getAnalysisFiles}}{\emph{Id}, \emph{queryPars=\{`Limit': `100'}, \emph{`SortBy': `Id'}, \emph{`SortDir': `Asc'}, \emph{`Offset': `0'\}}}{}
Returns a list of File object for the Analysis with id  = Id
\begin{quote}\begin{description}
\item[{Parameters}] \leavevmode\begin{itemize}
\item {} 
\textbf{Id} -- The id of the analysis.

\item {} 
\textbf{queryPars} -- An (optional) object of type QueryParameters for custom sorting and filtering

\end{itemize}

\end{description}\end{quote}

\end{fulllineitems}

\index{getAvailableGenomes() (BaseSpacePy.api.BaseSpaceAPI.BaseSpaceAPI method)}

\begin{fulllineitems}
\phantomsection\label{Available modules:BaseSpacePy.api.BaseSpaceAPI.BaseSpaceAPI.getAvailableGenomes}\pysiglinewithargsret{\bfcode{getAvailableGenomes}}{\emph{queryPars=\{`Limit': `100'}, \emph{`SortBy': `Id'}, \emph{`SortDir': `Asc'}, \emph{`Offset': `0'\}}}{}
Returns a list of all available genomes
\begin{quote}\begin{description}
\item[{Parameters}] \leavevmode
\textbf{queryPars} -- An (optional) object of type QueryParameters for custom sorting and filtering

\end{description}\end{quote}

\end{fulllineitems}

\index{getCoverageMetaInfo() (BaseSpacePy.api.BaseSpaceAPI.BaseSpaceAPI method)}

\begin{fulllineitems}
\phantomsection\label{Available modules:BaseSpacePy.api.BaseSpaceAPI.BaseSpaceAPI.getCoverageMetaInfo}\pysiglinewithargsret{\bfcode{getCoverageMetaInfo}}{\emph{Id}, \emph{Chrom}}{}
Returns Metadata about coverage as a CoverageMetadata instance
\begin{quote}\begin{description}
\item[{Parameters}] \leavevmode\begin{itemize}
\item {} 
\textbf{Id} -- he Id of the Bam file

\item {} 
\textbf{Chrom} -- Chromosome to query

\end{itemize}

\end{description}\end{quote}

\end{fulllineitems}

\index{getFileById() (BaseSpacePy.api.BaseSpaceAPI.BaseSpaceAPI method)}

\begin{fulllineitems}
\phantomsection\label{Available modules:BaseSpacePy.api.BaseSpaceAPI.BaseSpaceAPI.getFileById}\pysiglinewithargsret{\bfcode{getFileById}}{\emph{Id}}{}
Returns a file object by Id
\begin{quote}\begin{description}
\item[{Parameters}] \leavevmode
\textbf{Id} -- The id of the file

\end{description}\end{quote}

\end{fulllineitems}

\index{getFilesBySample() (BaseSpacePy.api.BaseSpaceAPI.BaseSpaceAPI method)}

\begin{fulllineitems}
\phantomsection\label{Available modules:BaseSpacePy.api.BaseSpaceAPI.BaseSpaceAPI.getFilesBySample}\pysiglinewithargsret{\bfcode{getFilesBySample}}{\emph{Id}, \emph{queryPars=\{`Limit': `100'}, \emph{`SortBy': `Id'}, \emph{`SortDir': `Asc'}, \emph{`Offset': `0'\}}}{}
Returns a list of File objects associated with sample with Id
\begin{quote}\begin{description}
\item[{Parameters}] \leavevmode\begin{itemize}
\item {} 
\textbf{Id} -- A Sample id

\item {} 
\textbf{queryPars} -- An (optional) object of type QueryParameters for custom sorting and filtering

\end{itemize}

\end{description}\end{quote}

\end{fulllineitems}

\index{getGenomeById() (BaseSpacePy.api.BaseSpaceAPI.BaseSpaceAPI method)}

\begin{fulllineitems}
\phantomsection\label{Available modules:BaseSpacePy.api.BaseSpaceAPI.BaseSpaceAPI.getGenomeById}\pysiglinewithargsret{\bfcode{getGenomeById}}{\emph{Id}}{}
Returns an instance of Genome with the specified Id
\begin{quote}\begin{description}
\item[{Parameters}] \leavevmode
\textbf{Id} -- The genome id

\end{description}\end{quote}

\end{fulllineitems}

\index{getIntervalCoverage() (BaseSpacePy.api.BaseSpaceAPI.BaseSpaceAPI method)}

\begin{fulllineitems}
\phantomsection\label{Available modules:BaseSpacePy.api.BaseSpaceAPI.BaseSpaceAPI.getIntervalCoverage}\pysiglinewithargsret{\bfcode{getIntervalCoverage}}{\emph{Id}, \emph{Chrom}, \emph{StartPos=None}, \emph{EndPos=None}}{}
Mean coverage levels over a sequence interval
\begin{quote}\begin{description}
\item[{Parameters}] \leavevmode\begin{itemize}
\item {} 
\textbf{Id} -- Chromosome to query

\item {} 
\textbf{Chrom} -- The Id of the resource

\item {} 
\textbf{StartPos} -- Get coverage starting at this position. Default is 1

\item {} 
\textbf{EndPos} -- Get coverage up to and including this position. Default is StartPos + 1280

\end{itemize}

\end{description}\end{quote}

:return:CoverageResponse -- an instance of CoverageResponse

\end{fulllineitems}

\index{getProjectById() (BaseSpacePy.api.BaseSpaceAPI.BaseSpaceAPI method)}

\begin{fulllineitems}
\phantomsection\label{Available modules:BaseSpacePy.api.BaseSpaceAPI.BaseSpaceAPI.getProjectById}\pysiglinewithargsret{\bfcode{getProjectById}}{\emph{Id}}{}
Request a project object by Id
\begin{quote}\begin{description}
\item[{Parameters}] \leavevmode
\textbf{Id} -- The Id of the project

\end{description}\end{quote}

\end{fulllineitems}

\index{getProjectByUser() (BaseSpacePy.api.BaseSpaceAPI.BaseSpaceAPI method)}

\begin{fulllineitems}
\phantomsection\label{Available modules:BaseSpacePy.api.BaseSpaceAPI.BaseSpaceAPI.getProjectByUser}\pysiglinewithargsret{\bfcode{getProjectByUser}}{\emph{Id}, \emph{queryPars=\{`Limit': `100'}, \emph{`SortBy': `Id'}, \emph{`SortDir': `Asc'}, \emph{`Offset': `0'\}}}{}
Returns a list available projects for a User with the specified Id
\begin{quote}\begin{description}
\item[{Parameters}] \leavevmode\begin{itemize}
\item {} 
\textbf{Id} -- The id of the user

\item {} 
\textbf{qp} -- An (optional) object of type QueryParameters for custom sorting and filtering

\end{itemize}

\end{description}\end{quote}

\end{fulllineitems}

\index{getSampleById() (BaseSpacePy.api.BaseSpaceAPI.BaseSpaceAPI method)}

\begin{fulllineitems}
\phantomsection\label{Available modules:BaseSpacePy.api.BaseSpaceAPI.BaseSpaceAPI.getSampleById}\pysiglinewithargsret{\bfcode{getSampleById}}{\emph{Id}}{}
Returns a Sample object
\begin{quote}\begin{description}
\item[{Parameters}] \leavevmode
\textbf{Id} -- The id of the sample

\end{description}\end{quote}

\end{fulllineitems}

\index{getSamplesByProject() (BaseSpacePy.api.BaseSpaceAPI.BaseSpaceAPI method)}

\begin{fulllineitems}
\phantomsection\label{Available modules:BaseSpacePy.api.BaseSpaceAPI.BaseSpaceAPI.getSamplesByProject}\pysiglinewithargsret{\bfcode{getSamplesByProject}}{\emph{Id}, \emph{queryPars=\{`Limit': `100'}, \emph{`SortBy': `Id'}, \emph{`SortDir': `Asc'}, \emph{`Offset': `0'\}}}{}
Returns a list of samples associated with a project with Id
\begin{quote}\begin{description}
\item[{Parameters}] \leavevmode\begin{itemize}
\item {} 
\textbf{Id} -- The id of the project

\item {} 
\textbf{queryPars} -- An (optional) object of type QueryParameters for custom sorting and filtering

\end{itemize}

\end{description}\end{quote}

\end{fulllineitems}

\index{getServerUri() (BaseSpacePy.api.BaseSpaceAPI.BaseSpaceAPI method)}

\begin{fulllineitems}
\phantomsection\label{Available modules:BaseSpacePy.api.BaseSpaceAPI.BaseSpaceAPI.getServerUri}\pysiglinewithargsret{\bfcode{getServerUri}}{}{}
Returns the server uri used by this instance

\end{fulllineitems}

\index{getUserById() (BaseSpacePy.api.BaseSpaceAPI.BaseSpaceAPI method)}

\begin{fulllineitems}
\phantomsection\label{Available modules:BaseSpacePy.api.BaseSpaceAPI.BaseSpaceAPI.getUserById}\pysiglinewithargsret{\bfcode{getUserById}}{\emph{Id}}{}
Returns the User object corresponding to Id
\begin{quote}\begin{description}
\item[{Parameters}] \leavevmode
\textbf{Id} -- The Id of the user

\end{description}\end{quote}

\end{fulllineitems}

\index{getVariantMetadata() (BaseSpacePy.api.BaseSpaceAPI.BaseSpaceAPI method)}

\begin{fulllineitems}
\phantomsection\label{Available modules:BaseSpacePy.api.BaseSpaceAPI.BaseSpaceAPI.getVariantMetadata}\pysiglinewithargsret{\bfcode{getVariantMetadata}}{\emph{Id}, \emph{Format}}{}
Returns a VariantMetadata object for the variant file
\begin{quote}\begin{description}
\item[{Parameters}] \leavevmode\begin{itemize}
\item {} 
\textbf{Id} -- The Id of the VCF file

\item {} 
\textbf{Format} -- Set to `vcf' to get the results as lines in VCF format

\end{itemize}

\end{description}\end{quote}

\end{fulllineitems}

\index{largeFileDownload() (BaseSpacePy.api.BaseSpaceAPI.BaseSpaceAPI method)}

\begin{fulllineitems}
\phantomsection\label{Available modules:BaseSpacePy.api.BaseSpaceAPI.BaseSpaceAPI.largeFileDownload}\pysiglinewithargsret{\bfcode{largeFileDownload}}{}{}
Not yet implemented

\end{fulllineitems}

\index{markAnalysisState() (BaseSpacePy.api.BaseSpaceAPI.BaseSpaceAPI method)}

\begin{fulllineitems}
\phantomsection\label{Available modules:BaseSpacePy.api.BaseSpaceAPI.BaseSpaceAPI.markAnalysisState}\pysiglinewithargsret{\bfcode{markAnalysisState}}{\emph{Id}, \emph{Status}, \emph{Summary}}{}
Set the status of an Analysis object
\begin{quote}\begin{description}
\item[{Parameters}] \leavevmode\begin{itemize}
\item {} 
\textbf{Id} -- The id of the analysis

\item {} 
\textbf{Status} -- The status assignment string must

\item {} 
\textbf{Summary} -- The summary string

\end{itemize}

\end{description}\end{quote}

\end{fulllineitems}

\index{multipartFileUpload() (BaseSpacePy.api.BaseSpaceAPI.BaseSpaceAPI method)}

\begin{fulllineitems}
\phantomsection\label{Available modules:BaseSpacePy.api.BaseSpaceAPI.BaseSpaceAPI.multipartFileUpload}\pysiglinewithargsret{\bfcode{multipartFileUpload}}{\emph{Id}, \emph{localPath}, \emph{fileName}, \emph{directory}, \emph{contentType}, \emph{cpuCount=1}, \emph{splits=10}, \emph{partSize=10}}{}
Not Yet implemented
\begin{quote}\begin{description}
\item[{Parameters}] \leavevmode\begin{itemize}
\item {} 
\textbf{Id} -- 

\item {} 
\textbf{localPath} -- 

\item {} 
\textbf{fileName} -- 

\item {} 
\textbf{directory} -- 

\item {} 
\textbf{contentType} -- 

\item {} 
\textbf{cpuCount} -- 

\item {} 
\textbf{splits} -- 

\item {} 
\textbf{partSize} -- 

\end{itemize}

\end{description}\end{quote}

\end{fulllineitems}


\end{fulllineitems}



\section{Models}
\label{Available modules:models}

\subsection{AppLaunch}
\label{Available modules:applaunch}\index{AppLaunch (class in BaseSpacePy.model.AppLaunch)}

\begin{fulllineitems}
\phantomsection\label{Available modules:BaseSpacePy.model.AppLaunch.AppLaunch}\pysigline{\strong{class }\code{BaseSpacePy.model.AppLaunch.}\bfcode{AppLaunch}}
AppLaunch contains the data returned
\index{getLaunchType() (BaseSpacePy.model.AppLaunch.AppLaunch method)}

\begin{fulllineitems}
\phantomsection\label{Available modules:BaseSpacePy.model.AppLaunch.AppLaunch.getLaunchType}\pysiglinewithargsret{\bfcode{getLaunchType}}{}{}
Returns a list {[}\textless{}launch type name\textgreater{}, list of objects{]} where \textless{}launch type name\textgreater{} is one of Projects, Samples, Analyses

\end{fulllineitems}


\end{fulllineitems}



\subsection{Project}
\label{Available modules:project}\index{Project (class in BaseSpacePy.model.Project)}

\begin{fulllineitems}
\phantomsection\label{Available modules:BaseSpacePy.model.Project.Project}\pysigline{\strong{class }\code{BaseSpacePy.model.Project.}\bfcode{Project}}
Represents a BaseSpace Project object.
\index{createAnalysis() (BaseSpacePy.model.Project.Project method)}

\begin{fulllineitems}
\phantomsection\label{Available modules:BaseSpacePy.model.Project.Project.createAnalysis}\pysiglinewithargsret{\bfcode{createAnalysis}}{\emph{api}, \emph{name}, \emph{desc}}{}
Return a newly created Analysis object
\begin{quote}\begin{description}
\item[{Parameters}] \leavevmode\begin{itemize}
\item {} 
\textbf{api} -- An instance of BaseSpaceAPI

\item {} 
\textbf{name} -- The name of the analysis

\item {} 
\textbf{desc} -- A describtion of the analysis

\end{itemize}

\end{description}\end{quote}

\end{fulllineitems}

\index{getAccessStr() (BaseSpacePy.model.Project.Project method)}

\begin{fulllineitems}
\phantomsection\label{Available modules:BaseSpacePy.model.Project.Project.getAccessStr}\pysiglinewithargsret{\bfcode{getAccessStr}}{\emph{scope='write'}}{}
Returns the scope-string to used for requesting BaseSpace access to the object
\begin{quote}\begin{description}
\item[{Parameters}] \leavevmode
\textbf{scope} -- The scope-type that is request (write\textbar{}read)

\end{description}\end{quote}

\end{fulllineitems}

\index{getAnalyses() (BaseSpacePy.model.Project.Project method)}

\begin{fulllineitems}
\phantomsection\label{Available modules:BaseSpacePy.model.Project.Project.getAnalyses}\pysiglinewithargsret{\bfcode{getAnalyses}}{\emph{api}}{}
Returns a list of Analysis objects.
\begin{quote}\begin{description}
\item[{Parameters}] \leavevmode
\textbf{api} -- An instance of BaseSpaceAPI

\end{description}\end{quote}

\end{fulllineitems}

\index{getSamples() (BaseSpacePy.model.Project.Project method)}

\begin{fulllineitems}
\phantomsection\label{Available modules:BaseSpacePy.model.Project.Project.getSamples}\pysiglinewithargsret{\bfcode{getSamples}}{\emph{api}}{}
Returns a list of Sample objects.
\begin{quote}\begin{description}
\item[{Parameters}] \leavevmode
\textbf{api} -- An instance of BaseSpaceAPI

\end{description}\end{quote}

\end{fulllineitems}

\index{isInit() (BaseSpacePy.model.Project.Project method)}

\begin{fulllineitems}
\phantomsection\label{Available modules:BaseSpacePy.model.Project.Project.isInit}\pysiglinewithargsret{\bfcode{isInit}}{}{}
Is called to test if the Project instance has been initialized.
\begin{description}
\item[{Throws:}] \leavevmode
ModelNotInitializedException - Indicates the object has not been populated yet.

\end{description}

\end{fulllineitems}


\end{fulllineitems}



\subsection{Analysis}
\label{Available modules:analysis}\index{Analysis (class in BaseSpacePy.model.Analysis)}

\begin{fulllineitems}
\phantomsection\label{Available modules:BaseSpacePy.model.Analysis.Analysis}\pysigline{\strong{class }\code{BaseSpacePy.model.Analysis.}\bfcode{Analysis}}~\index{getAccessStr() (BaseSpacePy.model.Analysis.Analysis method)}

\begin{fulllineitems}
\phantomsection\label{Available modules:BaseSpacePy.model.Analysis.Analysis.getAccessStr}\pysiglinewithargsret{\bfcode{getAccessStr}}{\emph{scope='write'}}{}
Returns the scope-string to be used for requesting BaseSpace access to the object
\begin{quote}\begin{description}
\item[{Parameters}] \leavevmode
\textbf{scope} -- The scope-type that is request (write\textbar{}read)

\end{description}\end{quote}

\end{fulllineitems}

\index{getFiles() (BaseSpacePy.model.Analysis.Analysis method)}

\begin{fulllineitems}
\phantomsection\label{Available modules:BaseSpacePy.model.Analysis.Analysis.getFiles}\pysiglinewithargsret{\bfcode{getFiles}}{\emph{api}, \emph{myQp=\{\}}}{}
Returns a list of file objects
\begin{quote}\begin{description}
\item[{Parameters}] \leavevmode\begin{itemize}
\item {} 
\textbf{api} -- An instance of BaseSpaceAPI

\item {} 
\textbf{myQp} -- (Optional) QueryParameters for sorting and filtering the file list

\end{itemize}

\end{description}\end{quote}

\end{fulllineitems}

\index{isInit() (BaseSpacePy.model.Analysis.Analysis method)}

\begin{fulllineitems}
\phantomsection\label{Available modules:BaseSpacePy.model.Analysis.Analysis.isInit}\pysiglinewithargsret{\bfcode{isInit}}{}{}
Is called to test if the Project instance has been initialized
\begin{description}
\item[{Throws:}] \leavevmode
ModelNotInitializedException  - if the instance has not been populated.

\end{description}

\end{fulllineitems}

\index{setStatus() (BaseSpacePy.model.Analysis.Analysis method)}

\begin{fulllineitems}
\phantomsection\label{Available modules:BaseSpacePy.model.Analysis.Analysis.setStatus}\pysiglinewithargsret{\bfcode{setStatus}}{\emph{api}, \emph{Status}, \emph{Summary}}{}
Sets the status of the analysis (note: once set to `completed' or `aborted' no more work can be done to the instance)
\begin{quote}\begin{description}
\item[{Parameters}] \leavevmode\begin{itemize}
\item {} 
\textbf{api} -- An instance of BaseSpaceAPI

\item {} 
\textbf{Status} -- The status value, must be completed, aborted, working, or suspended

\item {} 
\textbf{Summary} -- The status summary

\end{itemize}

\end{description}\end{quote}

\end{fulllineitems}

\index{uploadFile() (BaseSpacePy.model.Analysis.Analysis method)}

\begin{fulllineitems}
\phantomsection\label{Available modules:BaseSpacePy.model.Analysis.Analysis.uploadFile}\pysiglinewithargsret{\bfcode{uploadFile}}{\emph{api}, \emph{localPath}, \emph{fileName}, \emph{directory}, \emph{contentType}}{}
Uploads a local file to the BaseSpace analysis
\begin{quote}\begin{description}
\item[{Parameters}] \leavevmode\begin{itemize}
\item {} 
\textbf{api} -- An instance of BaseSpaceAPI

\item {} 
\textbf{localPath} -- The local path of the file

\item {} 
\textbf{fileName} -- The filename

\item {} 
\textbf{directory} -- The remote directory to upload to

\item {} 
\textbf{contentType} -- The contentype of the file

\end{itemize}

\end{description}\end{quote}

\end{fulllineitems}

\index{uploadMultipartFile() (BaseSpacePy.model.Analysis.Analysis method)}

\begin{fulllineitems}
\phantomsection\label{Available modules:BaseSpacePy.model.Analysis.Analysis.uploadMultipartFile}\pysiglinewithargsret{\bfcode{uploadMultipartFile}}{}{}
NOT YET IMPLMENTED.

\end{fulllineitems}


\end{fulllineitems}



\subsection{Sample}
\label{Available modules:sample}\index{Sample (class in BaseSpacePy.model.Sample)}

\begin{fulllineitems}
\phantomsection\label{Available modules:BaseSpacePy.model.Sample.Sample}\pysigline{\strong{class }\code{BaseSpacePy.model.Sample.}\bfcode{Sample}}
Representation of a BaseSpace Sample object.
\index{getAccessStr() (BaseSpacePy.model.Sample.Sample method)}

\begin{fulllineitems}
\phantomsection\label{Available modules:BaseSpacePy.model.Sample.Sample.getAccessStr}\pysiglinewithargsret{\bfcode{getAccessStr}}{\emph{scope='write'}}{}
Returns the scope-string to used for requesting BaseSpace access to the sample.
\begin{quote}\begin{description}
\item[{Parameters}] \leavevmode
\textbf{scope} -- The scope type that is request (write\textbar{}read).

\end{description}\end{quote}

\end{fulllineitems}

\index{getFiles() (BaseSpacePy.model.Sample.Sample method)}

\begin{fulllineitems}
\phantomsection\label{Available modules:BaseSpacePy.model.Sample.Sample.getFiles}\pysiglinewithargsret{\bfcode{getFiles}}{\emph{api}, \emph{myQp=\{\}}}{}
Returns a list of File objects
\begin{quote}\begin{description}
\item[{Parameters}] \leavevmode\begin{itemize}
\item {} 
\textbf{api} -- A BaseSpaceAPI instance

\item {} 
\textbf{myQp} -- Query parameters to sort and filter the file list by.

\end{itemize}

\end{description}\end{quote}

\end{fulllineitems}

\index{isInit() (BaseSpacePy.model.Sample.Sample method)}

\begin{fulllineitems}
\phantomsection\label{Available modules:BaseSpacePy.model.Sample.Sample.isInit}\pysiglinewithargsret{\bfcode{isInit}}{}{}
Is called to test if the sample instance has been initialized.
\begin{description}
\item[{Throws:}] \leavevmode
ModelNotInitializedException - Indicated the Id variable is not set.

\end{description}

\end{fulllineitems}


\end{fulllineitems}



\subsection{File}
\label{Available modules:file}\index{File (class in BaseSpacePy.model.File)}

\begin{fulllineitems}
\phantomsection\label{Available modules:BaseSpacePy.model.File.File}\pysigline{\strong{class }\code{BaseSpacePy.model.File.}\bfcode{File}}
Represents a BaseSpace file object.
\index{downloadFile() (BaseSpacePy.model.File.File method)}

\begin{fulllineitems}
\phantomsection\label{Available modules:BaseSpacePy.model.File.File.downloadFile}\pysiglinewithargsret{\bfcode{downloadFile}}{\emph{api}, \emph{localDir}, \emph{range=}\optional{}}{}
Download the file object to the specified localDir or a byte range of the file, by specifying the 
start and stop byte in the range.
\begin{quote}\begin{description}
\item[{Parameters}] \leavevmode\begin{itemize}
\item {} 
\textbf{api} -- A BaseSpaceAPI with read access on the scope including the file object.

\item {} 
\textbf{loadlDir} -- The local directory to place the file in.

\item {} 
\textbf{range} -- Specify the start and stop byte of the file chunk that needs retrieved.

\end{itemize}

\end{description}\end{quote}

\end{fulllineitems}

\index{filterVariant() (BaseSpacePy.model.File.File method)}

\begin{fulllineitems}
\phantomsection\label{Available modules:BaseSpacePy.model.File.File.filterVariant}\pysiglinewithargsret{\bfcode{filterVariant}}{\emph{api}, \emph{Chrom}, \emph{StartPos}, \emph{EndPos}, \emph{q=None}}{}
Returns a list of Variant objects available in the specified region
\begin{quote}\begin{description}
\item[{Parameters}] \leavevmode\begin{itemize}
\item {} 
\textbf{api} -- An instance of BaseSpaceAPI

\item {} 
\textbf{Chrom} -- Chromosome as a string - for example `chr2'

\item {} 
\textbf{StartPos} -- The start position of region of interest as a string

\item {} 
\textbf{EndPos} -- The end position of region of interest as a string

\item {} 
\textbf{q} -- An instance of

\end{itemize}

\end{description}\end{quote}

\end{fulllineitems}

\index{getCoverageMeta() (BaseSpacePy.model.File.File method)}

\begin{fulllineitems}
\phantomsection\label{Available modules:BaseSpacePy.model.File.File.getCoverageMeta}\pysiglinewithargsret{\bfcode{getCoverageMeta}}{\emph{api}, \emph{Chrom}}{}
Return an object of CoverageMetadata for the selected region
\begin{quote}\begin{description}
\item[{Parameters}] \leavevmode\begin{itemize}
\item {} 
\textbf{api} -- An instance of BaseSpaceAPI.

\item {} 
\textbf{Chrom} -- The chromosome of interest.

\end{itemize}

\end{description}\end{quote}

\end{fulllineitems}

\index{getIntervalCoverage() (BaseSpacePy.model.File.File method)}

\begin{fulllineitems}
\phantomsection\label{Available modules:BaseSpacePy.model.File.File.getIntervalCoverage}\pysiglinewithargsret{\bfcode{getIntervalCoverage}}{\emph{api}, \emph{Chrom}, \emph{StartPos}, \emph{EndPos}}{}
Return a coverage object for the specified region and chromosome.
\begin{quote}\begin{description}
\item[{Parameters}] \leavevmode\begin{itemize}
\item {} 
\textbf{api} -- An instance of BaseSpaceAPI

\item {} 
\textbf{Chrom} -- Chromosome as a string - for example `chr2'

\item {} 
\textbf{StartPos} -- The start position of region of interest as a string

\item {} 
\textbf{EndPos} -- The end position of region of interest as a string

\end{itemize}

\end{description}\end{quote}

\end{fulllineitems}

\index{getVariantMeta() (BaseSpacePy.model.File.File method)}

\begin{fulllineitems}
\phantomsection\label{Available modules:BaseSpacePy.model.File.File.getVariantMeta}\pysiglinewithargsret{\bfcode{getVariantMeta}}{\emph{api}}{}
Return the the meta info for a VCF file as a VariantInfo object
\begin{quote}\begin{description}
\item[{Parameters}] \leavevmode
\textbf{api} -- An instance of BaseSpaceAPI

\end{description}\end{quote}

\end{fulllineitems}

\index{isInit() (BaseSpacePy.model.File.File method)}

\begin{fulllineitems}
\phantomsection\label{Available modules:BaseSpacePy.model.File.File.isInit}\pysiglinewithargsret{\bfcode{isInit}}{}{}
Is called to test if the File instance has been initialized.
\begin{description}
\item[{Throws:}] \leavevmode
ModelNotInitializedException if the instance has not been populated yet.

\end{description}

\end{fulllineitems}

\index{isValidFileOption() (BaseSpacePy.model.File.File method)}

\begin{fulllineitems}
\phantomsection\label{Available modules:BaseSpacePy.model.File.File.isValidFileOption}\pysiglinewithargsret{\bfcode{isValidFileOption}}{\emph{filetype}}{}
Is called to test if the File instance is matches the filtype parameter
\begin{quote}\begin{description}
\item[{Parameters}] \leavevmode
\textbf{filetype} -- The filetype for coverage or variant requests

\end{description}\end{quote}

\end{fulllineitems}


\end{fulllineitems}



\subsection{QueryParameters}
\label{Available modules:queryparameters}\index{QueryParameters (class in BaseSpacePy.model.QueryParameters)}

\begin{fulllineitems}
\phantomsection\label{Available modules:BaseSpacePy.model.QueryParameters.QueryParameters}\pysiglinewithargsret{\strong{class }\code{BaseSpacePy.model.QueryParameters.}\bfcode{QueryParameters}}{\emph{pars=\{\}, required={[}'SortBy', `Offset', `Limit', `SortDir'{]}}}{}
The QueryParameters class can be passed as an optional arguments for a specific sorting of list-responses (such as lists of sample, analysis, or variants)

\end{fulllineitems}



\chapter{Indices and tables}
\label{index:indices-and-tables}\begin{itemize}
\item {} 
\emph{genindex}

\item {} 
\emph{modindex}

\item {} 
\emph{search}

\end{itemize}



\renewcommand{\indexname}{Index}
\printindex
\end{document}
